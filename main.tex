\documentclass{article}

% -----------------------
% BASIC PACKAGES & SETUP
% -----------------------
\usepackage[utf8]{inputenc}
\usepackage[margin=1in]{geometry}
\usepackage{setspace}
\usepackage[parfill]{parskip}     % blank line between paragraphs
\setlength{\parindent}{15pt}
\usepackage{indentfirst}
\usepackage{ragged2e}

% -----------------------
% MATH & SYMBOLS
% -----------------------
\usepackage{amsmath}
\usepackage{amssymb}

% -----------------------
% GRAPHICS & TABLES
% -----------------------
\usepackage{graphicx}
\usepackage{adjustbox}
\usepackage{rotating}
\usepackage{booktabs}
\usepackage{float}
\usepackage{pdflscape}
\DeclareGraphicsExtensions{.pdf,.png,.jpg}

% -----------------------
% CAPTIONS
% -----------------------
\usepackage{caption}
\usepackage{subcaption}

% -----------------------
% HYPERLINKS
% -----------------------
\usepackage[colorlinks=true,
            urlcolor=blue,
            linkcolor=blue,
            citecolor=blue]{hyperref}

% -----------------------
% COMMENTS + PDF TOOLS
% -----------------------
\usepackage{comment}
\usepackage{pdfpages}

% -----------------------
% BIBLIOGRAPHY (natbib)
% -----------------------
\usepackage[square,sort,comma,numbers]{natbib}
\bibliographystyle{agsm}
\bibpunct{(}{)}{;}{a}{,}{,}
\usepackage{url}
\pdfgentounicode=1


\title{The Effects of 287(g) Agreements on Immigrant's Mobility Decisions}
\author{Mena Kiser}
\date{\today}

\begin{document}
\doublespacing
\maketitle

\begin{abstract}
    287(g) agreements create partnerships between state and local agencies and ICE, allowing for deputized officers to perform some of ICE duties; this program led to deportations of immigrants of Hispanic origin burgeoning between 2011-2014. Out of fear of racial discrimination, Hispanic non-citizens may be more prone to moving out of counties with active 287(g) agreements. I explore this through a staggered difference-in-difference design evaluating the effect of living in a county with an active 287(g) agreement among the population targeted by this policy. As counties can volunteer into this agreement, I address selection concerns by applying a propensity score weights using parameters determinant of a county being exposed to the program at any point during our period of observation.
\end{abstract}

\section{Introduction}

This study exploits the staggered timing of activating and terminating a 287(g) agreements, a policy establishing partnerships between state and local enforcement agents and ICE for identifying non-citizens, to evaluate how immigration enforcement shapes immigrant settlement patterns. This paper combines newly reconstructed administrative data on 287(g) agreements with large-scale microdata from the ACS to implement a series of regression and staggered difference-in-differences designs. First, I exploit annual variation in whether a locality operates an active 287(g) agreement to estimate the effect of enforcement exposure on the total size of the targeted immigrant population at the puma–year level. To isolate the dynamic effects of activating or terminating enforcement agreements, I use a staggered difference-in-differences framework that separately identifies the effects of gaining and losing a 287(g) agreement, supplemented with event-study analyses to evaluate parallel trends. Across all designs, I pre-specify a narrow, policy-relevant target population constructed from TRAC-derived deportee characteristics, and I benchmark the results against a placebo population unaffected by deportation risk. I also evaluate an adjacent population that, though not directly targeted, might have been affected by the policy through racial profiling. These complementary empirical strategies allow for credible identification of how localized enforcement environments influence immigrant inflows and the evolving demographic presence of targeted groups.

Across specifications, the results reveal a consistent pattern: 287(g) agreements dampen the presence of the targeted immigrant population. Activation of a 287(g) agreement is associated with declines in the targeted foreign-born, low-skill, young Hispanic male population and, in some cases, modest declines in in-migration probabilities; however, these effects are often imprecise or modest in magnitude, aligning with the limited sensitivity of ACS stocks to short-run outflows. In contrast, the termination of a 287(g) agreement yields large and statistically robust increases in the target population—on the order of 15–20 percent overall, and substantially larger for recently arrived immigrants, those with low English proficiency, and immigrants without children. Event-study patterns further show clear post-termination reversals in population trajectories, with no corresponding movements for placebo groups. These findings underscore that localized immigration enforcement meaningfully shapes where targeted immigrants choose to reside and that the relaxation of enforcement generates rapid demographic recovery in formerly sanctioned areas.


\section{Literature review}

A large literature examines the consequences of increased enforcement on immigrants behavior and interactions with public agencies. \cite{wong_287g_2012} documents how 287(g) jurisdictions experience heightened police–immigrant contact, often producing racialized enforcement and reductions in immigrants’ willingness to report crime. \cite{baumer_federal_local_2023} documents how 287(g) task force agreements increased the likelihood that Latinos experiencing violent victimization. \cite{alsan_fear_2024} and \cite{watson_inside_2014} find that programs increasing risk of deportations reduced immigrants participation federal safety net programs. Other work highlights effects on health-care utilization and birth outcomes, as pregnant women and mixed-status families reduce interactions with public institutions (e.g. \cite{rhodes_impact_2015}; \cite{tome_heightened_2021}). Collectively, this body of research suggests that 287(g) alters immigrants’ daily decision-making, shaping local institutions and behavior even among those not directly subject to removal.

Studies consistently show that local immigration enforcement reshapes socioeconomic behavior and labor market outcomes. Increased deportation risk consistently depresses labor-market participation and wage income \citep{east_labor_2023}. Conversely, reductions in deportation risk generate substantial behavioral and economic responses: DACA increased high-school completion, college enrollment and other human capital investments \cite{kuka_human_2020}. Evaluating 287(g) specifically, \cite{shrestha_effects_2024} finds that county-level enactments of the program reduced the total number of businesses. \cite{ifft_is_2022} finds that 287(g) agreements lead to increased farm labor costs leading to technology investment and automation that insufficiently offset labor costs. By focusing on mobility changes, we can obtain additional evidence on the mechanism driving these changes.

Migrants may respond to interior enforcement by relocating, avoiding, or exiting jurisdictions with increased deportation risk. Using ACS and administrative enforcement data, \cite{leerkes_borders_2012} find that stricter interior enforcement, including 287(g), reduces in-migration and can increase out-migration among unauthorized or recently arrived immigrants. \cite{asad_hiding_2019} show that deportation risk reorders settlement patterns, producing “legal status geographies” in which immigrant families avoid counties where enforcement is activated. Policies that reduce deportation risk show the opposite effect, for instance DACA, increase residential stability and cross-county migration, reflecting lower mobility costs \citep{kiser_daca_2024}. This literature therefore establishes mobility as a clear behavioral margins through which immigrants respond to shifts in local deportation risk—whether through 287(g) activation, other enforcement initiatives, or protective policies.

\cite{parrado_immigration_2012}, the closest study to my setup, uses a difference-in-differences approach to estimate how newly enacted 287(g) agreements from 2007-2009, alongside recession-driven employment shocks, affected the size of the Mexican foreign-born population and employment outcomes for low-skilled natives. Parrado documents consistent declines in the Mexican foreign-born population but emphasizes substantial geographic heterogeneity across jurisdictions. My study complements and extends this work by examining a longer window of 287(g) changes (seven years rather than three) during a period of greater macroeconomic stability; leveraging newly reconstructed administrative data; separating the effects of gaining versus losing an agreement, identifying asymmetric population responses; integrating event studies identification of dynamic effects; and incorporating benchmarking against both placebo and adjacent at-risk populations. These differences provide a more comprehensive picture of how local immigration enforcement shapes immigrant settlement patterns over time and across policy transitions. 

\section{287(g) Background}
% 287g information
The 287(g) program establishes partnerships between state and local enforcement agencies and ICE, allowing for local officers to exercise ICE duties. Agencies can voluntarily request participation in the program and ICE ultimately decides which agencies are included in the program and enters negotiations for an agreement. Agreements are negotiated between DHS and local agencies and supervised by ICE, establishing delegation of authority to a determined number of officers. After an agreement expires, DHS is not obligated to renew it. Not all agreements include a specific expiration date, and once an agreement is entered into, it may be terminated at any time by either party.

This policy was officially enacted in 1996 as part of the Illegal Immigration Reform and Immigrant Responsibility Act (IIRIRA); however, no agencies joined the program until 2002 \citep{mpi2011}. The number of participating agencies varies depending on changes to standard templates of agreements and differences in recruiting and funding. Between 2006 and 2009, at least 56 agencies joined the program, partially due to the simplification of the contract negotiating process. In September of 2025, DHS reported over 1,000 current agreements, most of which were entered starting 2019 \citep{web:ice287}. Changes in recruiting, funding, and ICE and DHS priorities for identifying undocumented immigrants may change the program's intensity.

This program can take three different models (and hybrids of these) based on the needs, capacity, and interest of the agency and ICE, we can continuously observe only one type, jail enforcement modality, for the full extent of our period of observation (2013-2019)\footnote{\cite{mccann_analysis_2024} finds that the “jail enforcement” model endows more power than those operating under the “warrant service” model and only few agreements contain enforcement or encounter reporting requirements. This suggests that focusing on this modality can provide valuable insights on a major and modality and provides details on the nuance on how the policy is implemented.}. Under this type of agreement local officers can interrogate and place detainers (requests to maintain in custody for up to 48 hours) for suspected noncitizens who have been arrested. The task force model allows local officers to interrogate suspected noncitizens encounter in every day activities was rescinded starting in 2013\footnote{The decision to rescind this model was largely due to concerns of racial discrimination.} and reinstated in 2025--restricting our period of observation to 2013-2019 allows to obviate the effects of the termination of this modality and to focus on the predominant jail modality. The warrant service officer model, under which local officers to receive ICE training to execute immigration warrants, was first introduced in May 2019. We define our treatment as having an active 287(g) agreement in an individual locality as being exposed to any model of 287(g), understanding that jail enforcement modality is the predominant model being captured during this period. 

\section{Data}
The main datasets used in this paper come from the ICE websites and the American Community Survey (ACS). 

Obtaining a yearly list of active 287(g) agreements has posed an issue to other researchers (e.g. \cite{parrado_immigration_2012}). Through the Internet Archive's Wayback Machine \citeyear{wayback}, I can extract snapshots from ICE websites listing agencies with current 287(g) agreements, retrievable starting 2011. All lists include the name of the agency, the date it was signed, and the type of agreement entered. I then match this to the only ICE list listing signed 287(g) covering multiple years (2012, 2013, 2019) and find a 100\% match in agency names \citep{web:iceweb}. I match agencies to their local geographic area using a Bureau of Justice Statistics (\citeyear{data:leaxwalk}) crosswalk and then use a Geocorr crosswalk \citep{missouri_census_data_center_geocorr_2018} to obtain a match to a PUMA (2010 version). For each year, I can observe if a given locality had an active agreement during the first quarter of the calendar year\footnote{\cite{east_labor_2023} uses January to define enactment of Secure communities in a locality, I expand this to the first quarter to allow for the sparseness in retrieval}. 

I obtain details on the demographics of those targeted by 287(g) from the Transactional Records Access Clearinghouse (\citeyear{trac24}), an organization distributing statistics obtained from government agencies through the Freedom of Information Act (FOIA). From reports on ICE removals from 2013--2019 initiated with a 287(g) Program apprehension, summarized in Table \ref{tab:removals}, we can see that 76\% of the deportees were of ages 18--39, 98\% had citizenship in a Latin American country, and 96\% were male. We use these characteristics to identify the population targeted by this policy: males, Latin American country of origin, ages 18-39, foreign-born, non-citizen. We can also observe yearly trends in removals, as illustrated in \ref{tab:removals}, and we can see that during our period of observation 2013 removals reached to 12,000. Though this number is small, studies have shown how the fear induced by this policy is salient among Hispanic communities, and leads to changes in labor market decisions \citep{alsan_fear_2024}; furthermore, we restrict to a narrow population sector especially targeted by this policy.

Individual characteristics are extracted from the 2013-2019 American Community Survey (ACS), obtained through the Integrated Public Use Microdata Series (IPUMS) \citep{data:acs}. The ACS is a yearly cross-sectional, one percent, annual survey of households in the United States. Through this survey we can identify foreign-born respondents using place of birth and citizenship status; however, this survey does not ask about current legal status or status at entry. I specify the targeted population using the predominant characteristics of removed immigrants (males, Latin American country of origin, ages 18-39, foreign-born, non-citizen) and further restrict this sample to those not eligible to DACA, following \cite{kiser_daca_2024} identifications. To evaluate mobility patterns, I focus on a especially mobile population: not currently married, also ruling out citizenship eligibility through a spouse, those with less than five years in the country, and low-skill individual (educational attainment of High School degree or less). The ACS defines Public Use Microdata Areas (PUMAs), non-overlapping, statistical geographic areas that partition each state or equivalent entity into geographic areas containing no fewer than 100,000 people each--thus this is never missing for any individual. We use pumas as our geographic unit of analysis as we are able to identify treatment at this level. To evaluate puma-year population changes, I aggregate population by adding the population size using sampling weights, the resulting sample in structured at the puma-year level and contains totaled population for each target category and relevant covariates.

In Table \ref{tab:balance}, we can observe characteristics of the targeted population across never treated pumas and pumas that were ever treated. The target population in treated and never treated pumas have statistically insignificant differences in migpuma and state mobility, and movers from abroad. Individuals in treated pumas are similar in age and in share of the likelihood of being white, work similar weeks a year, and have similar income. However, we see some small but significant differences in other demographics and education characteristics signaling those in treated pumas have more children by 4 p.p., are more likely to have low English level by 7 p.p, and are slightly less likely to be in school. We see small but significant differences in employment covariates: the treated are employed at a slightly higher rate by 5 p.p, and work about 2 weekly hours more. Likewise, they have small but signficant differences in housing covariates: the treated are less likely to own a home (by 3 p.p.), perceive slightly lower rent (by \$69) and have slightly lower mortgage prices (by \$110). Overall, this signals that the targeted population across pumas with different treatment status have small but significant differences in demographics, employment, and housing characteristics. I will control for differences in the puma-year level for age, race, and homeownership.XX

During our period of observation (2013--2019), we can observe pumas that gain treatment (an active 287(g) agreement), lose an agreement, that never receive an agreement, or always have an active agreement. A total of 118 pumas are treated at any point between 2013--2019, of these, 26 are always treated, 73 gain treatment, and 21 lose treatment. XXX

\section{Empirical design}

Our main sample of observation spans from 2013-2019, covers pumas in states with any 287(g) active agreement during this period and with a non-zero target population. I focus on a population specially targeted and most responsive to policy changes, as detailed in the previous section, this includes immigrants with the following characteristics: males, ages 18-39, born in a Latin American country, not a U.S. citizen, that have spent 5 years or less in the country, that have a high school degree or less, and not currently married. The yearly total targeted population in the US in a treated puma is in the range of 200,000; while the average total population is between 400 and 500 (Figure \ref{fig:totpop_yr}). Focusing on a small population allows us to clearly identify changes that may drive changes in a more broadly defined sectors. XX 

To identify the effects of this policy on changes in the targeted population, I start with a regression analysis evaluating the effect of having an active 287(g) agreement in any given year on an puma's log of target population. I detail this design in Equation \ref{eq:regpop}, in which the $Y$, outcome variable, is the log target population currently residing in puma \textit{p} during year \textit{y}. The treatment is defined by having an active agreement in the puma of residence during by the first quarter of the survey year. I also include puma-year controls $X_{py}$ (log totals of population in different age distributions, race categories, high school attainment, in school status, and that are homeowners), puma fixed effects $W_{p}$ and year fixed effects $Z_{y}$. I cluster standard errors at the puma-year level. The coefficient of interest ($\beta_{1}$) measures the log change in the total target population in a puma when an active agreement is in place, relative to the same puma in untreated years and relative to contemporaneous changes in other pumas, holding area-year controls and fixed effects constant. This setup focuses on in-migration, by evaluating treatment in the \textit{current} puma of residence, rather than out-migration because it is possible a part of this population is leaving the country alltogether thus we would not be able to identify if they lived in treated puma the previous year.
\begin{equation}
    \label{eq:regpop}
    Y_{py} = \beta_{0} + \beta_{1}\text{active agreement}_{py} + \alpha_{1}X_{py} +  \alpha_{2}W_{p} + \alpha_{3}Z_{y} + \epsilon_{py} 
\end{equation}

To identify the effect of gaining and losing a 287(g) agreement in your local area, we implement a staggered difference-in-difference approach detailed in Equation \ref{eq:joindid}. The $\text{gains treatment}_{p}$ ($\text{loses treatment}_{p}$ ) indicator is assigned at the puma level and the $\text{post gain}_{py}$ ($\text{post loss}_{py}$) indicator is a dummy for the focus year being greater or equal than the year in which the treatment was gained (lost) in the focus puma. $\beta'_{1}$ captures the effect of gaining 287(g) treatment and $\beta'_{2}$ captures the effect of losing treatment. The comparison group includes the never treated; those that will lose treatment, but before they lose it; and those that will gain treatment, but before they gain it. We anticipate the effect of gaining treatment ($\beta'_{1}$) to differ in direction (sign) from the effect of losing treatment ($\beta'_{2}$): gaining treatment increases the risk of deportation and may drive the target population away from this area, while losing treatment decreases the risk of deportation and may make the location more appealing to immigrants mobile immigrants. Furthermore, we may expect the effect of gaining treatment to be more salient, as it is usually followed by news coverage more than an agreement termination would. This flexible equation allows us to identify asymmetric effects of 287(g) transitions, rather than the straightforward effect of having an agreement specified in Equation \ref{eq:regpop}. The identifying assumption expects that treated pumas, had not gained or lost treatment, would have experienced similar trends in totals of target population as never treated pumas. I evaluate this parallel trends assumption through an event study design by including dummies for pre and post for gaining an agreement and pre and post dummies for losing an agreement in the same specification. 
 \begin{equation}
    \label{eq:joindid}
    \begin{split}
    Y_{py} = \beta'_{0} + \beta'_{1}*\text{post gain}_{py} + \beta'_{2}\text{post loss}_{py} +  \alpha'_{1}W_{p} + 
    \alpha'_{1}X_{p} +  \alpha'_{2}W_{p} + \alpha'_{3}Z_{y} + \epsilon_{py} 
    \end{split}
\end{equation}

Though the main analysis focuses on the population directly targeted by 287(g) enforcement, I also examine two comparison groups. First, I construct a placebo population that should be entirely unaffected by the policy: U.S.-born, non-Hispanic white citizens with comparable non-immigration characteristics (males, ages 18–39, with a high school degree or less, and not currently married). Because this group is not at risk of deportation and does not match the demographic profile of individuals typically subject to immigration enforcement, they should not experience either direct effects or profiling. Estimating effects for this placebo group provides a benchmark for detecting spurious associations. Second, to assess potential spillover effects, I examine a population that is not deportable but may still be indirectly exposed to enforcement-related behaviors: U.S. citizens of Hispanic origin with the same demographic characteristics, either U.S.-born or naturalized prior to the start of the sample. I exclude individuals who naturalized during the study period, as naturalization decisions may respond endogenously to enforcement \citep{amuedo2020}. Although this spillover group is not targeted by 287(g), they may still modify relocation decisions in response to heightened scrutiny or local attitudes toward Hispanics \citep{baumer_federal_local_2023}. Comparing their responses to those of the placebo and treated groups helps distinguish direct enforcement effects from broader changes in the local environment.
  
\section{Results}

In Table \ref{tab:regpop} we evaluate the effect of having an active 287(g) agreement on the size of the target population (Columns 1-2) and placebo population (Columns 3-4) at the puma-year level (Equation \ref{eq:regpop}). In this design, I find that the size of the target population in a puma with an active 287(g) agreement declines by 0.0922 log points, or 8.8\% in population, and by 8.3\% when including controls at the puma-year level, both marginally significant at the 10 percent level. Meanwhile, the size of the placebo population increases at statistically insignificant rates: 12.8\%, and 13.4\% when including controls. Since we are focusing on a specially narrow target population, these coefficients translate translate to modest absolute changes in population size of ~70 people per puma. \citet{parrado_immigration_2012} report a difference-in-difference estimate of –6,488, most of which comes from changes in large metro areas, for the change in the foreign-born Mexican population at the metropolitan-area plus nonmetro PUMA aggregation level (a broader geography than a single PUMA). Although the paper does not explicitly state the exact comparison population used for scaling, this decline represents roughly a 7\% reduction relative to the pre-policy foreign-born Mexican population in areas that adopted 287(g). My regression findings align with the order of magnitude of this relative change though a careful examination of the difference-in-difference results is necessary.


%it would be nice to have an event study for this specification. The standard eventdd doesn't work until you split into gain & lose. But you could try including leads and lags of the treatment variable. i.e. create a lead1=1 you have 287(g) next year, and lead2=1 if you have a 298(g) in 2 years. Similarly, you can crate lag1 and lag2 if you had 287(g) last year or 2 years ago. Then you have 5 variables that trace out the event study -- lead2 is any effect showing up 2 years early, lead2+lead1 is effect 1 year early, lead2+lead1+treatment is effect in first treatment year, and so on (so you have to add up the coefficients). This assumes that everything is symmetric, but it is still useful.

I next hone into the effects of gaining and losing a 287(g) agreement by implementing the difference-in-difference design in Equation \ref{eq:joindid}, finding large but marginally significant results. In Table \ref{tab:didpop}, the target population decreases by 20.4\% (Column 2) when a puma gains a 287(g) agreement, significant at the 10 percent level, which translate to a decrease of 167 people relative to the mean puma targeted population size among the untreated. This is distinguishable from the effect in the placebo population, for which we find an imprecise effect of 8.2\% (Column 4). Losing a 287(g) agreement, with positive changes in the target population (6.0\%) though much smaller and with large standard errors, while the corresponding effect on the placebo population is negative and statistically insignificant though larger in magnitude (12.7\%). These patterns are consistent with intuitive expectations: when a locality activates a 287(g) agreement, increased deportation salience may reduce in-migration and/or increase out-migration among the targeted population, generating population declines. Conversely, when a locality loses a 287(g) agreement, reduced enforcement risk should encourage in-migration or slow out-migration, and our estimates align with this intuition. As previously mentioned, it is expected for the event of gaining an agreement to be more salient than losing an agreement due to media coverage or immediate enforcement changes, we see these reflected in the magnitude of our effects though these have large standard errors.

To assess the parallel trends assumption, I begin with an event–study design that estimates the dynamic effects of gaining and losing 287(g) agreements in the same specification with dummies for each pre and post period for each type of event.\footnote{The regression includes both gaining and losing events simultaneously; for clarity, I plot them separately.} Panel (a) shows that prior to gaining a 287(g) agreement, the size of the target population is gradually and slightly increasing, this may indicate that localities that enacted the policy were experiencing growth in the target population in the years prior to the enactment. After the agreement is enacted, the target population declines more sharply beginning one year post-treatment, reaches its largest drop two years after, suggesting a modest decline in population growth. The placebo population (Panel (c)) displays flat trends suggesting that before and after the reference year, the population size was larger but changes were negligible. Panel (b) presents the event study for losing an agreement. In the pre-period, population was consistently lower than in the reference period and stayed within a close range, after the enactment of the policy, the population size is smaller than in the reference period but larger than in the pre-period for the first two years of treatment, in the third year there is a slight increase in population size. We can see that there is no identifiable pattern in the lost of an agreement aligning with a reduction in enforcement risk leading to the return of in-migrants, suggesting that the noticeable risk of deportation in an area about to lose an agreement is not much different from an area that has recently lost it. For the placebo population (Panel (c)), we see that population size remained stable in the pre-period and dropped slightly during the year of enactment but later returned to the regular average; these trends are not substantially different from the effect in the target population suggesting both groups are unaffected by losing an agreement. 

%I think this is not the right interpretation. It looks like there was some increase in the target population that preceeded losing the 287(g), so a bit of a pre-trend. Quesiton is whether this reflected declining enforcement leading up to actually losing the 287(g). or is it sign of a violation of parallel trends?
%the event studies are underwhelming. One thing is that you are probably going too far before/after, given that you only have 7 years of data. Try shortening the window to something like -3 to +2? (where the last point on either end accumulates, so -3 or less, +2 or more). This will also get rid of some of the huge CIs at the extremes, which will help with the scale of the y-axis (currently has a big range to accomodate some of the big CIs).
%again, this will look much better if you do -3 to +2. There is a pretty clear pre-trend going on, but you might imagine there is a reason for this -- enforcement peters out before the actual end of the formal agreement?


\subsection{Heterogeneity and spillovers}
Panel A of Table \ref{tab:didhet} estimates Equation \ref{eq:joindid} using the log population size of several subsectors of the target group: Mexican-born immigrants, immigrants with poor English proficiency, long-term immigrants (at least 5 years in the U.S.), immigrants without children, and a non-Hispanic comparison group. Across these subgroups, the estimated effects of gaining a 287(g) agreement remain large but statistically insignificant. The exception is those with poor English level (Column 4), where the point estimate is large (-0.725 log points, -51.6\%) and significant at the 5 percent level. As our target population is already restricted to recent immigrants, the strong negative results for those with poor English (likely to have been in the country for less time) suggests this policy affects where recent arrivals decide to settle. In contrast, the effect of losing a 287(g) agreement are smaller and statistically insignificant, with the exception of the effect on long-term immigrants for which we see an effect of 17.8\%; however, this subgroup sees a positive and imprecise effect of gaining enforcement. 

Panel B repeats the exercise for a spillover group: Hispanic U.S. citizens (either U.S. born or naturalized before the start of the period) with similar demographic characteristics. Although not directly targeted for deportation, this group may still experience racial profiling or include recently naturalized former members of the target population. Across spillover populations, gaining a 287(g) treatment shows no consistent effects, except for a large positive increase in long-term immigrants (0.70 log points), though this population is very small. Losing treatment similarly shows limited impacts, with the only meaningful change being a decline in adults with no children of –0.15 log points, significant at the 5 percent level.
%these results are kind of all over the map. Some of the populations are fairly small which may explain some of the instability. Any reason why these should show up mostly on the gain variable, while targeted group more shows up on the lose variable? I'm not sure we learn much from the spillover population. Maybe panel B should be the placebo population -- to show that you don't find effects in these subpopulations in the placebo?

\subsection{Propensity Score}
As Table \ref{tab:balance} showed small though statistically significant differences of the targeted population in the treated and untreated pumas and to address selection of pumas into the 287(g) program, I weight pumas using propensity scores. These are calculated through a logit estimation using 2013 (start of period of observation) puma characteristics and weighting using the total population of target immigrants to estimate the probability of having an active 287(g) during this period. 
The characteristics include 2013 total adult population and, to match on composition of localities, using 2013 shares of
\begin{spacing}{1}
\begin{enumerate}
    \item Target population
    \item Foreign born
    \item Young (18-39)
    \item Marital status
    \item Red state
    \item Texas
    \item State treatment
    \item Race
    \item Ethnicity
    \item Education
    \item English level
\end{enumerate}
\end{spacing}

Though all other specification exclude pumas that always have an active 287(g) during 2013-2019, as they are ultimately absorbed by puma fixed effects, I include them for finding propensity scores as they provide information of the type of localities with a 287(g) agreement. 

I apply these propensity scores to our difference-in-difference specification (Equation \ref{eq:joindid}) in Table \ref{tab:didpopprop}. For the target population, we see that the propensity weighting, increases the the effect of gaining treatment, and dampens the effect of losing treatment. Gaining a 287(g) agreement can be associated with a reduction of approximate 23–24\% in population size. In contrast, losing a 287(g) agreement has no meaningful effect, with estimates near zero, and less than half in magnitude than in Table \ref{tab:didpop}. For the placebo U.S.-born population, neither gaining nor losing treatment produces statistically meaningful changes: the estimated effects (about +8–15\% for gains and –4–6\% for losses) are imprecise. These results suggest that the effect of gaining an agreement is much more salient than losing an agreement and can be associated with modest declines in population, consistent with unweighted results.

\section{Discussion}



\bibliography{bib_287}


\section{Tables and figures}    
\begin{spacing}{1}
    

%%%%%% TABLES

%% table 1
\begin{table}[H]
\centering
\caption{ICE Removals from a 287(g) apprehension}
\label{tab:removals}
\adjustbox{ width=0.7\linewidth}{
    \begin{tabular}{rllll}
\toprule
\toprule
\multicolumn{1}{l}{Characteristics} & \multicolumn{1}{r}{Removals} & \multicolumn{1}{r}{Share ($\%$)}   \\
\midrule
\multicolumn{1}{l}{Gender} & &   \\
Male & 87,923 & 95.87   \\
Female & 3,790 & 4.13   \\
\multicolumn{1}{l}{Age group} & &   \\
0-17 & 82 & 0.09   \\
18-24 & 17,911 & 19.53   \\
25-29 & 21,588 & 23.54   \\
30-34 & 19,568 & 21.34   \\
35-39 & 14,631 & 15.95   \\
40-44 & 9,040 & 9.86   \\
45-49 & 4,985 & 5.44   \\
50-54 & 2,384 & 2.60   \\
55-59 & 1,004 & 1.09   \\
60-64 & 365 & 0.40   \\
65-69 & 115 & 0.13   \\
70-74 & 32 & 0.03   \\
75+ & 9 & 0.01   \\
\multicolumn{1}{l}{Latin American Citizen} & 90,235 & 98.39   \\
\multicolumn{1}{l}{Citizenship} & &   \\
Mexico & 68,038 & 74.18   \\
Guatemala & 8,042 & 8.77   \\
Honduras & 6,618 & 7.22   \\
El Salvador & 5,300 & 5.78   \\
Nicaragua & 329 & 0.36   \\
Brazil & 306 & 0.33   \\
\multicolumn{1}{l}{Level of Conviction} & &   \\
No Conviction & 16,270 & 17.74   \\
Level 1 Crime & 25,620 & 27.93   \\
Level 2 Crime & 9,065 & 9.88   \\
Level 3 Crime & 40,759 & 44.44   \\
\bottomrule
\bottomrule
\end{tabular}
    
}

\justifying

\begin{spacing}{1}
\begin{footnotesize}
\noindent \textit{Notes:} This table summarizes characteristics of foreign citizens removed by ICE following a 287(g) apprehension from 2013-2019. Latin American citizenship is identified through the top 50 reported countries. Citizenship reports only the leading 5 countries of citizenship. Source: TRAC Immigration Tools \citep{trac24}.
\end{footnotesize}
\end{spacing}
\end{table}


%% table 2
\newpage
\begin{table}[H]
\begin{center}

\caption{Characteristics of target population in treated and untreated pumas}
\label{tab:balance}
\adjustbox{ width=0.8\linewidth}{
    \begin{tabular}{lccc}
\toprule
\toprule
 & \multicolumn{3}{c}{Target population} & \multicolumn{3}{c}{Migpuma}  \\
 & Treated & Untreated & Difference   \\
 & (1) & (2) & (3)  \\
\midrule 
 \textbf{Mobility} & & &   \\
 Moved migpuma  & 0.12  & 0.13  & -0.00 \\
 & (0.01) & (0.00) & (0.01)\\
 Moved state  & 0.10  & 0.09  & 0.01 \\
 & (0.00) & (0.00) & (0.01)\\
 Moved from abroad  & 0.09  & 0.07  & 0.02*** \\
 & (0.00) & (0.00) & (0.01)\\
 \textbf{Demographics and education} & & &   \\
 Age  & 26.31  & 26.14  & 0.17 \\
 & (0.09) & (0.08) & (0.12)\\
 Number of children  & 0.21  & 0.21  & -0.00 \\
 & (0.01) & (0.01) & (0.01)\\
 Race: White  & 0.64  & 0.53  & 0.10*** \\
 & (0.01) & (0.01) & (0.01)\\
 Race: Black  & 0.01  & 0.02  & -0.01*** \\
 & (0.00) & (0.00) & (0.00)\\
 High School  & 0.47  & 0.45  & 0.01 \\
 & (0.01) & (0.01) & (0.01)\\
 Poor English  & 0.69  & 0.67  & 0.02* \\
 & (0.01) & (0.01) & (0.01)\\
 In School  & 0.09  & 0.08  & 0.01 \\
 & (0.00) & (0.00) & (0.01)\\
 \textbf{Employment and housing} & & &   \\
 Employed  & 0.84  & 0.84  & 0.01 \\
 & (0.01) & (0.00) & (0.01)\\
 Weeks worked  & 40.01  & 41.39  & -1.37 \\
 & (0.78) & (0.65) & (1.02)\\
 Usual weekly hours worked  & 35.19  & 35.30  & -0.11 \\
 & (0.24) & (0.20) & (0.31)\\
 Wage income  & 21,500.92  & 21,852.45  & -351.53 \\
 & (251.16) & (237.65) & (345.80)\\
 Owns a home  & 0.16  & 0.15  & 0.01 \\
 & (0.01) & (0.00) & (0.01)\\
 Rent price  & 1,008.73  & 1,035.85  & -27.11** \\
 & (7.41) & (8.04) & (10.93)\\
 Mortgage price  & 872.33  & 910.53  & -38.20 \\
 & (31.73) & (27.31) & (41.88)\\
Sample size & 5,963 & 8,151 & \\
\bottomrule
\bottomrule
\\
\end{tabular}
}
\end{center}

\justifying

\begin{spacing}{1}
\begin{footnotesize}
\noindent \textit{Notes:} This table summarizes main characteristics the population targeted by 287(g): males, ages 18-39, born in a Latin American country, not a U.S. citizen, that have spent 10 years or less in the country, that have a high school degree or less, and not currently married. Column 1 summarizes average characteristics for those in pumas ever treated from 2013-2019 and Column 2 for those in never treated pumas. Column 3 summarizes the average of those in treated areas minus the average of those in untreated areas. p $<$ 0.01 ***, p $<$ 0.05 **, p $<$0.1 *. 
\end{footnotesize}
\end{spacing}
\end{table}



%% table 4
\newpage
\begin{table}[H]
\centering
\caption{Net effect of an active 287(g) on log total targeted and log total placebo populations}
\label{tab:regpop}
\adjustbox{ width=0.9\linewidth}{
    \begin{tabular}{lcccc}
\toprule
\toprule
 & \multicolumn{2}{c}{Target population} & \multicolumn{2}{c}{Placebo population}  \\
 Log population & (1) & (2)  & (3) & (4)  \\
\midrule 
 Treated migpuma & -0.1357** & -0.1603** & -0.0170 & -0.0456 \\
 & (0.0634) & (0.0650) & (0.0567) & (0.0528) \\
\\
 Controls &  & X &  & X \\
 \textit{R2} & 0.9219 & 0.9236 & 0.9498 & 0.9515  \\
 Untreated pop size & 1,013 & 1,013 & 7,319 & 7,319  \\
Sample Size & 1,505 & 1,505 & 1,505 & 1,505  \\
\bottomrule
\bottomrule
\\
\end{tabular}
}

\justifying
\begin{spacing}{1}
\begin{footnotesize}


\noindent \textit{Notes:} This table summarizes the results from the regression design detailed in Equation \ref{eq:regpop}. Columns 1-2 evaluates the effect of having an active 287(g) agreement on the log total targeted population and Columns 3-4 on the log total placebo population, for comparison. Columns 2 and 3 incorporate controls at the puma-year level: log totals of population in different age distributions, race categories, high school attainment, in school status, and that are homeowners. The unit of observation is the puma-year level. All specifications include puma and year fixed effects and robust standard errors. p $<$ 0.01 ***, p $<$ 0.05 **, p $<$0.1 *. 

\end{footnotesize}
\end{spacing}
\end{table}





\newpage
\begin{table}[H]
\centering
\caption{Effect of gaining and losing a 287(g) agreement on the log total targeted and the log total placebo populations}
\label{tab:didpop}
\adjustbox{ width=0.9\linewidth}{
    \begin{tabular}{lcccc}
\toprule
\toprule
 & \multicolumn{2}{c}{Target population} & \multicolumn{2}{c}{Placebo population}  \\
Log population & (1) & (2)  & (3) & (4) \\
\midrule 
 Gain treatment & -0.2050 & -0.2284* & 0.0518 & 0.0787 \\
 & (0.1249) & (0.1285) & (0.1021) & (0.1012) \\
 Lose treatment & 0.0696 & 0.0583 & -0.1340 & -0.1354 \\
 & (0.0617) & (0.0602) & (0.1538) & (0.1509) \\
\\
 Controls &  & X &  & X \\
 \textit{R2} & 0.5949 & 0.6038 & 0.7401 & 0.7455  \\
 Untreated pop size & 832 & 832 & 1,483 & 1,483  \\
Sample Size & 3,487 & 3,487 & 3,487 & 3,487  \\
\bottomrule
\bottomrule
\\
\end{tabular}
}

\justifying
\begin{spacing}{1}
\begin{footnotesize}


\noindent \textit{Notes:} This table summarizes the results from the difference-in-difference detailed in Equation \ref{eq:joindid}. Columns 1-2 evaluates the effect of gaining a 287(g) agreement and losing a 287(g) agreement on the log total targeted population and Columns 3-4 on the log total placebo population, for comparison. Columns 2 and 3 incorporate controls at the puma-year level. The unit of observation is the puma-year level. All specifications include puma and year fixed effects and robust standard errors. p $<$ 0.01 ***, p $<$ 0.05 **, p $<$0.1 *. 

\end{footnotesize}
\end{spacing}
\end{table}




\begin{landscape}
\begin{table}[t]
\centering
\caption{Effect of gaining and losing a 287(g) agreement on the log total of sectors of the targeted population and the log total of sectors of the placebo}
\label{tab:didhet}
\adjustbox{ width=0.8\linewidth}{
    \begin{tabular}{lcccc}
\toprule
\toprule
 & \multicolumn{7}{c}{Panel A: Target population} \\
\midrule
 & Baseline & Mexican & Poor English & New Immigrant & No children & Non-Hispanic \\
Log population & (1) & (2)  & (3) & (4) & (5) & (6) \\
\midrule 
 Gain treatment & -0.0599 & -0.0389 & -0.0462 & 0.2998 &  &  \\
 & (0.0725) & (0.7040) & (0.1446) & (0.2636) & () & () \\
 Lose treatment & 0.1759*** & 0.0779 & 0.2997*** & 0.5388*** &  &  \\
 & (0.0629) & (0.2199) & (0.0896) & (0.1864) & () & () \\
\\
 Controls &  & X &  & X \\
 \textit{R2} & 0.9333 & 0.6921 & 0.7742 & 0.7178 &  &     \\
 Untreated mean & 7.8908 & 6.0503 & 7.2443 & 5.7511 &  &   \\
Sample Size & 2,115 & 2,115 & 2,115 & 2,115 &  &   \\
\midrule
\midrule
 & \multicolumn{7}{c}{Panel B: Spillover population} \\
 & Baseline & Mexican & Poor English & New Immigrant & No children & Non-Hispanic \\
Log population & (1) & (2)  & (3) & (4) & (5) & (6) \\
 Gain treatment & 0.2772* & 1.2077*** & -0.2509 & -1.1275** & 0.2360 & -0.1788 \\
 & (0.1561) & (0.4614) & (0.3505) & (0.5230) & (0.1605) & (0.1978) \\
 Lose treatment & -0.0472 & 0.2033 & -0.0721 & 0.6577 & 0.0299 & -0.2100* \\
 & (0.1103) & (0.2024) & (0.2591) & (0.5670) & (0.1125) & (0.1195) \\
\\
 Controls & X & X & X & X & X & X \\
 \textit{R2} & 0.8446 & 0.7625 & 0.7505 & 0.6310 &  &     \\
 Untreated mean & 6.4636 & 4.6037 & 3.8017 & 1.8135 &  &   \\
Sample Size & 2,115 & 2,115 & 2,115 & 2,115 &  &   \\
\bottomrule
\bottomrule
\\
\end{tabular}
}

\justifying
\begin{spacing}{1}
\begin{footnotesize}

\noindent \textit{Notes:} This table summarizes the results from the difference-in-difference detailed in Equation \ref{eq:joindid} on sizes of various population sectors. Panel A evaluates the effect of gaining and losing a 287(g) agreement on the size of different and similar sectors of the target population. Columns 1 represents the baseline, corresponding to Column 2 from Table \ref{tab:didpop}. Column 2 restricts the evaluated population to those from Mexican origin, Column 3 to those with poor English level, Column 4 to those with up to two years in the country, Column 5 to those with no children, and Column 6 evaluates non-Hispanic immigrants that share all other characteristics with the target population. Panel B repeats the top panel for the spillover population population: foreign-born U.S. citizens from Hispanic descent that have naturalized before 2013, with similar non-immigration characteristics to the targeted (males, ages 18-39, with a high school degree or less, not currently married).  p $<$ 0.01 ***, p $<$ 0.05 **, p $<$0.1 *.
\end{footnotesize}
\end{spacing}
\end{table}
\end{landscape}


%% table 2
\newpage
\begin{table}[H]
\begin{center}

\caption{Propensity score weighted characteristics of target population in treated and untreated pumas}
\label{tab:balance}
\adjustbox{ width=0.8\linewidth}{
    \begin{tabular}{lccc}
\toprule
\toprule
 & \multicolumn{3}{c}{Target population} \\
 & Treated & Untreated & Difference   \\
 & (1) & (2) & (3)  \\
\midrule 
 \textbf{Mobility} & & &   \\
 Moved migpuma  & 0.15  & 0.17  & -0.02 \\
 & (0.01) & (0.01) & (0.01)\\
 Moved state  & 0.13  & 0.14  & -0.01 \\
 & (0.01) & (0.01) & (0.01)\\
 Moved from abroad  & 0.11  & 0.11  & -0.00 \\
 & (0.01) & (0.01) & (0.01)\\
 \textbf{Demographics and education} & & &   \\
 Age  & 25.74  & 25.87  & -0.13 \\
 & (0.13) & (0.17) & (0.21)\\
 Has a child  & 0.10  & 0.09  & 0.01 \\
 & (0.01) & (0.01) & (0.01)\\
 Race: White  & 0.61  & 0.64  & -0.03* \\
 & (0.01) & (0.01) & (0.02)\\
 Race: Black  & 0.01  & 0.01  & -0.00 \\
 & (0.00) & (0.00) & (0.00)\\
 High School  & 0.42  & 0.45  & -0.03 \\
 & (0.01) & (0.01) & (0.02)\\
 Poor English  & 0.74  & 0.68  & 0.05*** \\
 & (0.01) & (0.01) & (0.02)\\
 In School  & 0.08  & 0.10  & -0.02** \\
 & (0.01) & (0.01) & (0.01)\\
 \textbf{Employment and housing} & & &   \\
 Employed  & 0.86  & 0.83  & 0.03** \\
 & (0.01) & (0.01) & (0.01)\\
 Weeks worked  & 39.44  & 41.23  & -1.79 \\
 & (1.15) & (1.46) & (1.86)\\
 Usual weekly hours worked  & 35.76  & 34.94  & 0.83 \\
 & (0.33) & (0.42) & (0.53)\\
 Wage income  & 20,780.39  & 21,929.83  & -1,149.44* \\
 & (356.09) & (484.36) & (601.25)\\
 Owns a home  & 0.14  & 0.20  & -0.06*** \\
 & (0.01) & (0.01) & (0.01)\\
 Rent price  & 938.16  & 989.10  & -50.95*** \\
 & (9.94) & (14.68) & (17.73)\\
 Mortgage price  & 752.75  & 723.96  & 28.80 \\
 & (43.28) & (41.48) & (59.99)\\
\\
Sample size & 2,747 & 6,307 & \\
Total population & 437,418 & 1,105,425 & \\
\bottomrule
\bottomrule
\\
\end{tabular}
}
\end{center}

\justifying

\begin{spacing}{1}
\begin{footnotesize}
\noindent \textit{Notes:} This table summarizes main characteristics (weighted by propensity score) the population targeted by 287(g): males, ages 18-39, born in a Latin American country, not a U.S. citizen, that have spent 10 years or less in the country, that have a high school degree or less, and not currently married. Column 1 summarizes average characteristics for those in pumas ever treated from 2013-2019 and Column 2 for those in never treated pumas. Column 3 summarizes the average of those in treated areas minus the average of those in untreated areas. p $<$ 0.01 ***, p $<$ 0.05 **, p $<$0.1 *. 
\end{footnotesize}
\end{spacing}
\end{table}


\begin{table}[h!]
\centering
\caption{Effect of gaining and losing a 287(g) agreement on the log total targeted and the log total placebo populations with propensity score weighting}
\label{tab:didpopprop}
\adjustbox{ width=0.7\linewidth}{
    \begin{tabular}{lcccc}
\toprule
\toprule
 & \multicolumn{2}{c}{Target population} & \multicolumn{2}{c}{Placebo population}  \\
Log population & (1) & (2)  & (3) & (4) \\
\midrule 
 Gain treatment & 0.0145 & 0.0005 & 0.1877 & 0.1550 \\
 & (0.1681) & (0.1680) & (0.1305) & (0.1296) \\
 Lose treatment & -0.0095 & 0.0023 & -0.1278*** & -0.1155** \\
 & (0.0722) & (0.0718) & (0.0457) & (0.0451) \\
\\
 Controls &  & X &  & X \\
 \textit{R2} & 0.4980 & 0.5025 & 0.7589 & 0.7648  \\
 Untreated mean & 6.2827 & 6.2827 & 7.0167 & 7.0167  \\
Sample Size & 7,905 & 7,905 & 7,905 & 7,905  \\
\bottomrule
\bottomrule
\\
\end{tabular}
}

\justifying
\begin{spacing}{1}
\begin{footnotesize}


\noindent \textit{Notes:} This table summarizes the results from the difference-in-difference detailed in Equation \ref{eq:joindid}, when including propensity scores weights. Columns 1-2 evaluates the effect of gaining a 287(g) agreement and losing a 287(g) agreement on the log total targeted population and Columns 3-4 on the log total placebo population, for comparison. Columns 2 and 3 incorporate controls at the puma-year level. The unit of observation is the puma-year level. All specifications include puma and year fixed effects and robust standard errors. p $<$ 0.01 ***, p $<$ 0.05 **, p $<$0.1 *. 

\end{footnotesize}
\end{spacing}
\end{table}





%%%%% FIGURES


\begin{figure}[H]
\centering
\caption{Yearly trends in 287(g) removals}
\label{fig:agreement_count}
\begin{center}
\adjustbox{width=\linewidth}{
    \includegraphics[width=\linewidth]{output/final/deportations.png}
}
\end{center}
\justifying
\begin{spacing}{1}
\begin{footnotesize}
\noindent \textit{Notes:} This graph summarizes yearly trends in ICE reomvals following a 287(g) apprehension from 2013-2019. Source: TRAC Immigration Tools \citep{trac24}.
\end{footnotesize}
\end{spacing}
\end{figure}



\begin{figure}[H]
\centering
\caption{Total target population}
\label{fig:totpop_yr}
\begin{center}
(a) Total target population
\adjustbox{width=\linewidth}{
    \includegraphics[width=0.7\linewidth]{output/final/total_target2_pop.png}
}
(a) Mean puma target population
\adjustbox{width=\linewidth}{
    \includegraphics[width=0.7\linewidth]{output/final/mean_target2_pop.png}
}
\end{center}
\justifying
\begin{spacing}{1}
\begin{footnotesize}
\noindent \textit{Notes:} This graph summarizes yearly trends in ICE reomvals following a 287(g) apprehension from 2013-2019. Source: TRAC Immigration Tools \citep{trac24}.
\end{footnotesize}
\end{spacing}
\end{figure}


%% figure 1

%% figure 2 
\begin{landscape}
\begin{figure}[t]
\centering
\caption{Event study of the effect of losing and gaining a 287(g) agreement on the log total targeted and placebo populations}
\label{fig:didpop}

\begin{minipage}{0.44\linewidth}
\centering
\adjustbox{width=\linewidth}{\includegraphics{output/final/logtargetpop_gain_estudy.png}}
\end{minipage}
\begin{minipage}{0.44\linewidth}
\centering
\adjustbox{width=\linewidth}{\includegraphics{output/final/logtargetpop_lost_estudy.png}}
\end{minipage}
\begin{minipage}{0.44\linewidth}
\centering
\adjustbox{width=\linewidth}{\includegraphics{output/final/logplacebopop_gain_estudy.png}}
\end{minipage}
\begin{minipage}{0.44\linewidth}
\centering
\adjustbox{width=\linewidth}{\includegraphics{output/final/logplacebopop_lost_estudy.png}}
\end{minipage}

\justifying
\begin{spacing}{1}
\begin{footnotesize}
\noindent \textit{Notes}: These event studies explore the effect of gaining and losing an agreement of the log of total population, using the specifications of Table \ref{tab:didpop}. Panel (a) plots the effect of gaining an agreement on the log of total target population and Panel (b), the effect of losing an agreement. Panel (c) plots the effect of gaining an agreement on the log of total placebo population and Panel (d), the effect of losing an agreement. All specifications include puma-year controls (log totals of population in different age distributions, race categories, high school attainment, in school status, and that are homeowners), puma and year fixed effects, and robust standard errors.
\end{footnotesize}
\end{spacing}

\end{figure}

\end{landscape}

\newpage
\begin{figure}[t]
\centering
\caption{Density of propensity score}
\label{fig:prop_score}
\begin{center}
\adjustbox{width=0.7\linewidth}{
    \includegraphics[width=\linewidth]{output/final/prop_score.png}
}
\end{center}
\justifying
\begin{spacing}{1}
\begin{footnotesize}
\noindent \textit{Notes:} This graph shows the density of the probability of a puma been treated at any point from 2013-2019 for the treatment group, the control, and the control upon weighting on propensity scores.
\end{footnotesize}
\end{spacing}
\end{figure}



\end{spacing}







\end{document}
