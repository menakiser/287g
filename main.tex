\documentclass{article}

% -----------------------
% BASIC PACKAGES & SETUP
% -----------------------
\usepackage[utf8]{inputenc}
\usepackage[margin=1in]{geometry}
\usepackage{setspace}
\usepackage[parfill]{parskip}     % blank line between paragraphs
\setlength{\parindent}{15pt}
\usepackage{indentfirst}
\usepackage{ragged2e}

% -----------------------
% MATH & SYMBOLS
% -----------------------
\usepackage{amsmath}
\usepackage{amssymb}

% -----------------------
% GRAPHICS & TABLES
% -----------------------
\usepackage{graphicx}
\usepackage{adjustbox}
\usepackage{rotating}
\usepackage{booktabs}
\usepackage{float}
\usepackage{pdflscape}
\DeclareGraphicsExtensions{.pdf,.png,.jpg}

% -----------------------
% CAPTIONS
% -----------------------
\usepackage{caption}
\usepackage{subcaption}

% -----------------------
% HYPERLINKS
% -----------------------
\usepackage[colorlinks=true,
            urlcolor=blue,
            linkcolor=blue,
            citecolor=blue]{hyperref}

% -----------------------
% COMMENTS + PDF TOOLS
% -----------------------
\usepackage{comment}
\usepackage{pdfpages}

% -----------------------
% BIBLIOGRAPHY (natbib)
% -----------------------
\usepackage[square,sort,comma,numbers]{natbib}
\bibliographystyle{agsm}
\bibpunct{(}{)}{;}{a}{,}{,}
\usepackage{url}
\pdfgentounicode=1


\title{The Effects of 287(g) Agreements on Immigrant's Mobility Decisions}
\author{Mena Kiser}
\date{\today}

\begin{document}
\doublespacing
\maketitle

\begin{abstract}
    287(g) agreements create partnerships between state and local agencies and ICE, allowing for deputized officers to perform some of ICE duties; this program led to deportations of immigrants of Hispanic origin burgeoning between 2011-2014. Out of fear of racial discrimination, Hispanic non-citizens may be more prone to moving out of counties with active 287(g) agreements. I explore this through a staggered difference-in-difference design evaluating the effect of living in a county with an active 287(g) agreement among the population targeted by this policy. As counties can volunteer into this agreement, I address selection concerns by applying a propensity score weights using parameters determinant of a county being exposed to the program at any point during our period of observation.
\end{abstract}

\section{Introduction}

This study exploits the staggered timing of activating and terminating a 287(g) agreements, a policy establishing partnerships between state and local enforcement agents and ICE for identifying non-citizens, to evaluate how immigration enforcement shapes immigrant settlement patterns. This paper combines newly reconstructed administrative data on 287(g) agreements with large-scale microdata from the ACS to implement a series of regression and staggered difference-in-differences designs. First, I exploit annual variation in whether a locality operates an active 287(g) agreement to estimate the effect of enforcement exposure on the total size of the targeted immigrant population at the puma–year level. To isolate the dynamic effects of activating or terminating enforcement agreements, I use a staggered difference-in-differences framework that separately identifies the effects of gaining and losing a 287(g) agreement, supplemented with event-study analyses to evaluate parallel trends. Across all designs, I pre-specify a narrow, policy-relevant target population constructed from TRAC-derived deportee characteristics, and I benchmark the results against a placebo population unaffected by deportation risk. I also evaluate an adjacent population that, though not directly targeted, might have been affected by the policy through racial profiling. These complementary empirical strategies allow for credible identification of how localized enforcement environments influence immigrant inflows and the evolving demographic presence of targeted groups.

Across specifications, the results reveal a consistent pattern: 287(g) agreements dampen the presence of the targeted immigrant population. Activation of a 287(g) agreement is associated with declines in the targeted foreign-born, low-skill, young Hispanic male population and, in some cases, modest declines in in-migration probabilities; however, these effects are often imprecise or modest in magnitude, aligning with the limited sensitivity of ACS stocks to short-run outflows. In contrast, the termination of a 287(g) agreement yields large and statistically robust increases in the target population—on the order of 15–20 percent overall, and substantially larger for recently arrived immigrants, those with low English proficiency, and immigrants without children. Event-study patterns further show clear post-termination reversals in population trajectories, with no corresponding movements for placebo groups. These findings underscore that localized immigration enforcement meaningfully shapes where targeted immigrants choose to reside and that the relaxation of enforcement generates rapid demographic recovery in formerly sanctioned areas.


\section{Literature review}

A large literature examines the consequences of increased enforcement on immigrants behavior and interactions with public agencies. \cite{wong_287g_2012} documents how 287(g) jurisdictions experience heightened police–immigrant contact, often producing racialized enforcement and reductions in immigrants’ willingness to report crime. \cite{baumer_federal_local_2023} documents how 287(g) task force agreements increased the likelihood that Latinos experiencing violent victimization. \cite{alsan_fear_2024} and \cite{watson_inside_2014} find that programs increasing risk of deportations reduced immigrants participation federal safety net programs. Other work highlights effects on health-care utilization and birth outcomes, as pregnant women and mixed-status families reduce interactions with public institutions (e.g. \cite{rhodes_impact_2015}; \cite{tome_heightened_2021}). Collectively, this body of research suggests that 287(g) alters immigrants’ daily decision-making, shaping local institutions and behavior even among those not directly subject to removal.

Studies consistently show that local immigration enforcement reshapes socioeconomic behavior and labor market outcomes. Increased deportation risk consistently depresses labor-market participation and wage income \citep{east_labor_2023}. Conversely, reductions in deportation risk generate substantial behavioral and economic responses: DACA increased high-school completion, college enrollment and other human capital investments \cite{kuka_human_2020}. Evaluating 287(g) specifically, \cite{shrestha_effects_2024} finds that county-level enactments of the program reduced the total number of businesses. \cite{ifft_is_2022} finds that 287(g) agreements lead to increased farm labor costs leading to technology investment and automation that insufficiently offset labor costs. By focusing on mobility changes, we can obtain additional evidence on the mechanism driving these changes.

Migrants may respond to interior enforcement by relocating, avoiding, or exiting jurisdictions with increased deportation risk. Using ACS and administrative enforcement data, \cite{leerkes_borders_2012} find that stricter interior enforcement, including 287(g), reduces in-migration and can increase out-migration among unauthorized or recently arrived immigrants. \cite{asad_hiding_2019} show that deportation risk reorders settlement patterns, producing “legal status geographies” in which immigrant families avoid counties where enforcement is activated. Policies that reduce deportation risk show the opposite effect, for instance DACA, increase residential stability and cross-county migration, reflecting lower mobility costs \citep{kiser_daca_2024}. This literature therefore establishes mobility as a clear behavioral margins through which immigrants respond to shifts in local deportation risk—whether through 287(g) activation, other enforcement initiatives, or protective policies.

\cite{parrado_immigration_2012}, the closest study to my setup, uses a difference-in-differences approach to estimate how newly enacted 287(g) agreements from 2007-2009, alongside recession-driven employment shocks, affected the size of the Mexican foreign-born population and employment outcomes for low-skilled natives. Parrado documents consistent declines in the Mexican foreign-born population but emphasizes substantial geographic heterogeneity across jurisdictions. My study complements and extends this work by examining a longer window of 287(g) changes (seven years rather than three) during a period of greater macroeconomic stability; leveraging newly reconstructed administrative data; separating the effects of gaining versus losing an agreement, identifying asymmetric population responses; integrating event studies identification of dynamic effects; and incorporating benchmarking against both placebo and adjacent at-risk populations. These differences provide a more comprehensive picture of how local immigration enforcement shapes immigrant settlement patterns over time and across policy transitions. 

\section{287(g) Background}
% 287g information
The 287(g) program establishes partnerships between state and local enforcement agencies and ICE, allowing for local officers to exercise ICE duties. Agencies can voluntarily request participation in the program and ICE ultimately decides which agencies are included in the program and enters negotiations for an agreement. Agreements are negotiated between DHS and local agencies and supervised by ICE, establishing delegation of authority to a determined number of officers. After an agreement expires, DHS is not obligated to renew it. Not all agreements include a specific expiration date, and once an agreement is entered into, it may be terminated at any time by either party.

This policy was officially enacted in 1996 as part of the Illegal Immigration Reform and Immigrant Responsibility Act (IIRIRA); however, no agencies joined the program until 2002 \citep{mpi2011}. The number of participating agencies varies depending on changes to standard templates of agreements and differences in recruiting and funding. Between 2006 and 2009, at least 56 agencies joined the program, partially due to the simplification of the contract negotiating process. In September of 2025, DHS reported over 1,000 current agreements, most of which were entered starting 2019 \citep{web:ice287}. Changes in recruiting, funding, and ICE and DHS priorities for identifying undocumented immigrants may change the program's intensity.

This program can take three different models (and hybrids of these) based on the needs, capacity, and interest of the agency and ICE, we can continuously observe only one type, jail enforcement modality, for the full extent of our period of observation (2013-2019)\footnote{\cite{mccann_analysis_2024} finds that the “jail enforcement” model endows more power than those operating under the “warrant service” model and only few agreements contain enforcement or encounter reporting requirements. This suggests that focusing on this modality can provide valuable insights on a major and modality and provides details on the nuance on how the policy is implemented.}. Under this type of agreement local officers can interrogate and place detainers (requests to maintain in custody for up to 48 hours) for suspected noncitizens who have been arrested. The task force model allows local officers to interrogate suspected noncitizens encounter in every day activities was rescinded starting in 2013\footnote{The decision to rescind this model was largely due to concerns of racial discrimination.} and reinstated in 2025--restricting our period of observation to 2013-2019 allows to obviate the effects of the termination of this modality and to focus on the predominant jail modality. The warrant service officer model, under which local officers to receive ICE training to execute immigration warrants, was first introduced in May 2019. We define our treatment as having an active 287(g) agreement in an individual locality as being exposed to any model of 287(g), understanding that jail enforcement modality is the predominant model being captured during this period. 

\section{Data}
The main datasets used in this paper come from the ICE websites and the American Community Survey (ACS). 

Obtaining a yearly list of active 287(g) agreements has posed an issue to other researchers (e.g. \cite{parrado_immigration_2012}). Through the Internet Archive's Wayback Machine \citeyear{wayback}, I can extract snapshots from ICE websites listing agencies with current 287(g) agreements, retrievable starting 2011. All lists include the name of the agency, the date it was signed, and the type of agreement entered. I then match this to the only ICE list listing signed 287(g) covering multiple years (2012, 2013, 2019) and find a 100\% match in agency names \citep{web:iceweb}. I match agencies to their local geographic area using a Bureau of Justice Statistics (\citeyear{data:leaxwalk}) crosswalk and then use a Geocorr crosswalk \citep{missouri_census_data_center_geocorr_2018} to obtain a match to a PUMA (2010 version). For each year, I can observe if a given locality had an active agreement during the first quarter of the calendar year\footnote{\cite{east_labor_2023} uses January to define enactment of Secure communities in a locality, I expand this to the first quarter to allow for sparseness in retrieval.}. Following \cite{wang_health_2019}, I distinct between local agreements and state agreements (i.e. state-level departments of corrections) as they have different enforcement mechanisms and enforcement to outcomes of interest, I do this by including a control for state-level agreement and focusing on local agreements.
I obtain details on the demographics of those targeted by 287(g) from the Transactional Records Access Clearinghouse (\citeyear{trac24}), an organization distributing statistics obtained from government agencies through the Freedom of Information Act (FOIA). From reports on ICE removals from 2013--2019 initiated with a 287(g) Program apprehension, summarized in Table \ref{tab:removals}, we can see that 76\% of the deportees were of ages 18--39, 98\% had citizenship in a Latin American country, and 96\% were male. We use these characteristics to identify the population targeted by this policy: males, Latin American country of origin, ages 18-39, foreign-born, non-citizen. We can also observe yearly trends in removals, as illustrated in \ref{tab:removals}, and we can see that during our period of observation 2013 removals reached to 12,000. Though this number is small, studies have shown how the fear induced by this policy is salient among Hispanic communities, and leads to changes in labor market decisions \citep{alsan_fear_2024}; furthermore, we restrict to a narrow population sector especially targeted by this policy.

Individual characteristics are extracted from the 2013-2019 American Community Survey (ACS), obtained through the Integrated Public Use Microdata Series (IPUMS) \citep{data:acs}. The ACS is a yearly cross-sectional, one percent, annual survey of households in the United States. Through this survey we can identify foreign-born respondents using place of birth and citizenship status; however, this survey does not ask about current legal status or status at entry. I specify the targeted population using the predominant characteristics of removed immigrants (males, Latin American country of origin, ages 18-39, foreign-born, non-citizen) and further restrict this sample to those with a U.S. arrival after 2007 to rule out DACA eligibility. As we focus in migration patterns we focus on individuals mobile individuals: not currently married, also ruling out citizenship eligibility through a spouse, and those with less than ten years in the country, and low-skill individual (educational attainment of High School degree or less). The ACS defines Public Use Microdata Areas (PUMAs), non-overlapping, statistical geographic areas that partition each state or equivalent entity into geographic areas containing no fewer than 100,000 people each--thus this is never missing for any individual. Migration indicators provide information using migration PUMAs, named migpumas, a geographic unit that aggregates one or multiple PUMAs. We use migpuma as our geographic unit of analysis as we are able to identify treatment and mobility at this level. 

 In Table \ref{tab:balance}, we can observe characteristics of the targeted population across never treated migpumas and migpumas that were ever treated. The target population in treated and never treated migpumas have statistically insignificant differences in migpuma and state mobility, but treated migpumas are more likely to have movers from abroad by (2 percentage points). Individuals in treated migpumas are similar in demographics and educational level: they are close in age and race composition, and attain high school and are currently in school at similar rates. We observe a small but significant difference in English level, signaling those in treated migpumas are more likely to have low English level. They are also similar in employment covariates: employed at similar rates, similar total weeks worked, usual hours, and close wage income (though those in treated migpumas have lower income levels). They also have similar home ownership rates and mortgage price, but we can identify a small but significant difference of rent price. Overall, this signals that the targeted population in migpumas with different treatment status are similar in demographics, employment, and housing characteristics.

During our period of observation (2013--2019), we can observe migpumas that gain treatment (an active 287(g) agreement), lose an agreement, that never receive an agreement, or always have an active agreement. A total of 118 migpumas are treated at any point between 2013--2019, of these, 26 are always treated, 73 gain treatment, and 21 lose treatment. 

%emphasize that where equation 1 was estimated with individual-level data, equation 2 is estimated with aggregate data at the migpuma-year leavel. Also, here or in the data section you should say specifically how you estimated this, e.g. based on summing up the sampling weights across all people in the target population living in the migpuma.

\section{Empirical design}

Our main sample of observation spans from 2013-2019 and covers migpumas in states with any 287(g) active agreement during this period. I focus on a population specially targeted and more responsive to policy changes, as detailed in the previous section, this includes immigrants with the following characteristics: males, ages 18-39, born in a Latin American country, not a U.S. citizen, that have spent 10 years or less in the country, that have a high school degree or less, and not currently married.

To identify the effects of this policy on in-migration, I start with a regression analysis evaluating the effect of having an active 287(g) agreement in any given year on an individual's in migration decisions. I detail this design in Equation  \ref{eq:regpr},  in which the $Y$, outcome variables, is a dummy for having moved in the past year for an individual \textit{i}, currently residing in migpuma \textit{m}, during year \textit{y}. The treatment is defined by having an active agreement in the migpuma of residence during the focus year. I also include individual controls $X_{i}$ (age, race, high school attained, currently in school, poor level of English, and a dummy for owning a house) migpuma fixed effects $W_{m} $ and year fixed effects $Z_{y}$ and cluster standard errors at the migpuma and year level. The coefficient of interest $\beta_{1}$ informs us of the net effect of having an active agreement on the chances of having moved to the migpuma of observation. I focus on in-migration, rather than out-migration because it is possible some of these immigrants are leaving the country alltogether thus we would not be able to identify if they lived in treated migpuma the previous year.
\begin{equation}
    \label{eq:regpr}
    Y_{imy} = \beta_{0} + \beta_{1}\text{active agreement}_{my} + \alpha_{1}X_{i} +  \alpha_{2}W_{m} + \alpha_{3}Z_{y} + \epsilon_{my} 
\end{equation}

As evaluating the probability of having moved to a treated migpuma relies in the overall number and composition (newcomers and stayers) of target immigrants in the locality, it is possible $\beta_{1}$ in Equation  \ref{eq:regpr} captures an outcome based on an evolving sample. Thus, I turn to evaluating the effect of the treatment in the total target population in an area. In this specification, detailed in Equation \ref{eq:regpop}, the outcome is the log target population at the migpuma and year level. The treatment, migpuma fixed effects, and year fixed effects are defined the same as in the previous equation. We add controls at the migpuma-year level, instead of at the individual level, these controls include log totals of population in different age distributions, race categories, high school attainment, in school status, and that are homeowners. These controls mirror those in the previous specification and are included to capture differences across areas with aging population, and changing in racial and education composition, or difficulty of owning a home. The coefficient of interest ($\beta'_{1}$) measures the percent change in the total target population in a migpuma when an active agreement is in place, relative to the same migpuma in untreated years and relative to contemporaneous changes in other migpumas, holding area-year controls and fixed effects constant.
\begin{equation}
    \label{eq:regpop}
    Y_{my} = \beta'_{0} + \beta'_{1}\text{active agreement}_{my} + \alpha'_{1}X_{my} +  \alpha'_{2}W_{m} + \alpha'_{3}Z_{y} + \epsilon_{my} 
\end{equation}

To identify the effect of gaining and losing a 287(g) agreement in your local area, we implement a staggered difference-in-difference approach detailed in Equation \ref{eq:joindid}. The $\text{gains treatment}_{m}$ ($\text{loses treatment}_{m}$ ) indicator is assigned at the migpuma level and the $\text{post gain}_{my}$ ($\text{post loss}_{my}$) indicator is a dummy for the focus year being greater or equal than the year in which the treatment was gained (lost) in the focus migpuma. $\beta''_{1}$ captures the effect of gaining 287(g) treatment and $\beta''_{2}$ captures the effect of losing treatment. The comparison group includes the never treated; those that will lose treatment, but before they lose it; and those that will gain treatment, but before they gain it. This setup allows us to identify the effect of obtaining or losing the treatment allowing for both events to be different from each other.
%you could mention that equation 2 assumes that b1''=-b2'', so this spec is more flexible and allows for asymmetric impacts of gaining and losing 287(g). You might mention why these effects might be asymmetric.
Because ACS population counts are more sensitive to changes in inflows than to changes in outflows due to the sampling design, this specification is better powered to detect increases in the target population. The identifying assumption expects that treated migpumas, had not gained or lost treatment, would have experienced similar trends in totals of target population as never treated migpumas. I will evaluate this parallel trends assumption through an event study design. %you might say more here. are you estimating each event study controlling for the other event (e.g. when you estimate eventdd for gaining a 287(g), do you control for post_loss?).
 \begin{equation}
    \label{eq:joindid}
    \begin{split}
    Y_{my} = \beta''_{0} + \beta''_{1}*\text{post gain}_{my} + \beta''_{2}\text{post loss}_{my} +  \alpha''_{1}W_{m} + 
    \alpha''_{1}X_{m} +  \alpha''_{2}W_{m} + \alpha''_{3}Z_{y} + \epsilon_{my} 
    \end{split}
\end{equation}

Though we focus on a population specifically targeted by 287(g) deportations, I also evaluate effects on a placebo population that is not affected by this policy: U.S. born citizens of Hispanic origin with similar non-immigration characteristics (males, ages 18-39, with a high school degree or less, not currently married). Evaluating a placebo population would allow us to discern if effects in the total population are distinguishable from those that are not affected. As a way to evaluate spillover effects, I can also evaluate a population that \textit{should not} be affected by the policy: foreign-born U.S. citizens from Hispanic descent, with similar non-immigration characteristics (males, ages 18-39, with a high school degree or less, not currently married). It is worth noting that this population may have previously been part of the target population. Intensified enforcement can prompt some eligible immigrants to naturalize faster as a way to gain protection from deportation \citep{amuedo2020}.

%I'm confused about the distinction between the placebo population and the spillover population. are you thinking us born are completely unaffected, whereas foreign-born citizens might have some spillover? State this more clearly.
    
 \section{Results}

Table \ref{tab:regprob} evaluates Equation \ref{eq:regpr} in a sample restricted to the target population in Columns 1 and 2 and restricted to the placebo population in Columns 3-4. Through this setup we find that the target population in a migpuma with an active 287(g) agreement is more likely to have moved to their current place of residence within the last year by 2.6 percentage points (or 21.25\% more relative to untreated migpumas) this results remain within this level after controlling for individual covariates. In the placebo population we find a negligible effect of the treatment in the probability of having moved to the current migpuma. The saliency of these results indicate that the composition of the target population in treated migpuma is different from comparable untreated migpumas. The direction and magnitude of these results should be taken with caution as they are not accounting for an evolving population size within a migpuma. It is possible that treated localities are observing changes in the total target population that have lived in the locality for over a year, thus driving the probability of moving upwards. %this discussion is confusing. Try first reporting the results and emphasizing the unexpected sign. Then have a paragraph discussing why this might be misleading and how this suggests looking directly at the size of the target population. Then in a new paragraph you discuss table 4.

We evaluate this in Table \ref{tab:regpop}, in which we evaluate changes in the size of the target and placebo populations at the migpuma-year level (Equation \ref{eq:regpop}). In this design, I find that the size of the target population in a migpuma with an active 287(g) agreement declines by 11.8\% and by 12.6\% when including controls at the migpuma-year level, both significant at the 1 percent level. Meanwhile, the size of the placebo population decreases at much smaller and statistically insignificant rates: 1.6\% and 2.4\% when including controls. %This is an important result and one of your key findings. Say a bit more -- e.g. 12% decline in population seems huge, but need to remember that this is a relatively small targeted population. Are there any prior studies that estimate something similar that you could compare to?
%it would be nice to have an event study for this specification. The standard eventdd doesn't work until you split into gain & lose. But you could try including leads and lags of the treatment variable. i.e. create a lead1=1 you have 287(g) next year, and lead2=1 if you have a 298(g) in 2 years. Similarly, you can crate lag1 and lag2 if you had 287(g) last year or 2 years ago. Then you have 5 variables that trace out the event study -- lead2 is any effect showing up 2 years early, lead2+lead1 is effect 1 year early, lead2+lead1+treatment is effect in first treatment year, and so on (so you have to add up the coefficients). This assumes that everything is symmetric, but it is still useful.


I next hone into the effects of gaining and losing a 287(g) agreement by implementing the difference-in-difference design in Equation \ref{eq:joindid}. In Table \ref{tab:didpop}, the target population decreases by 5.8\% (Column 2) when a migpuma gains a 287(g) agreement, though this estimate is imprecise and similar in magnitude to the change observed for the placebo population, which declines by 7.4\% (Column 4). Losing a 287(g) agreement, however, is associated with a 19.2\% increase in the size of the target population (Column 2), significant at the 1 percent level, while the corresponding effect on the placebo population is negligible (–0.46\%) and statistically insignificant. These patterns are consistent with intuitive expectations: when a locality activates a 287(g) agreement, increased deportation salience may reduce in-migration and/or increase out-migration among the targeted population, generating population declines. Conversely, when a locality loses a 287(g) agreement, reduced enforcement risk should encourage in-migration or slow out-migration, and our estimates align with this intuition. Additionally, increases in population are more easily detected in this ACS-based design. Similar results are found by implementing propensity score weighting\footnote{I repeat this specification applying propensity score weighting in Table \ref{eq:didpopprop}, resulting in larger estimates for gaining treatment (still imprecise) and similar though slightly smaller estimates for losing treatment (Column 2). The placebo population perceive smaller estimates for the effect of gaining treatment and larger for losing treatment, both still statistically significant.}. % hold off on this
%But why is the effect of losing 287(g) so much larger? Is there a story for this?

To assess the parallel trends assumption, I begin with an event–study design that estimates the dynamic effects of gaining and losing 287(g) agreements.\footnote{The regression includes both gaining and losing events simultaneously; for clarity, I plot them separately.} Panel (a) shows that prior to gaining a 287(g) agreement, changes in the size of the target population are close to zero, though generally below the reference period (the year immediately before treatment). After the agreement is enacted, the target population declines more sharply beginning one year post-treatment, reaches its largest drop two years after, and returns toward zero by year three. The placebo population (Panel (c)) displays a similar pattern during the first few years: modest declines through two years post-treatment and continued downward movement in year three. %the event studies are underwhelming. One thing is that you are probably going too far before/after, given that you only have 7 years of data. Try shortening the window to something like -3 to +2? (where the last point on either end accumulates, so -3 or less, +2 or more). This will also get rid of some of the huge CIs at the extremes, which will help with the scale of the y-axis (currently has a big range to accomodate some of the big CIs).
%again, this will look much better if you do -3 to +2. There is a pretty clear pre-trend going on, but you might imagine there is a reason for this -- enforcement peters out before the actual end of the formal agreement?
Panel (b) presents the event study for losing an agreement. In the pre-period, estimates are consistently negative but move toward zero as the termination date approaches, indicating that areas about to lose 287(g) were already experiencing declines in the target population, though these declines were moderating over time. After the agreement ends, the coefficients shift to positive values and gradually increase. While some of the post-period estimates are imprecise, the overall pattern implies a reversal of the earlier decline, with population levels beginning to recover and ultimately surpassing the pre-termination trajectory. %I think this is not the right interpretation. It looks like there was some increase in the target population that preceeded losing the 287(g), so a bit of a pre-trend. Quesiton is whether this reflected declining enforcement leading up to actually losing the 287(g). or is it sign of a violation of parallel trends?

This pattern aligns with a reduction in enforcement risk leading to the return of in-migrants or a slowdown in out-migration. For the placebo population (Panel (c)), however, the event-study coefficients around the loss of a 287(g) agreement are small, statistically insignificant, and lack the clear pre-trend reversal seen in the target population, this reinforces the positive post-termination effects observed in the target group. 

\subsection{Heterogeneity and spillovers}
Panel A of Table 6 estimates Equation \ref{eq:joindid} using the log population size of several subsectors of the target group: Mexican-born immigrants, immigrants with poor English proficiency, recent arrivals ($\leqslant$ 2 years in the U.S.), immigrants without children, and a non-Hispanic comparison group. Across these subgroups, the estimated effects of gaining a 287(g) agreement remain small and statistically insignificant. The only exception is recent immigrants (Column 4), where the point estimate is large (+0.30 log points = +35\%) but imprecise. This pattern is consistent with the idea that ACS-based population counts are only weakly sensitive to short-run out-migration among targeted groups.
By contrast, losing a 287(g) agreement produces positive and statistically significant effects for several subgroups. The baseline effect (Column 1) indicates a +0.176 log-point increase (= +19.2\%) in the target population relative to untreated areas. The effects are even larger among groups more exposed to enforcement risk: +0.300 log points for immigrants with poor English (= +35\%), +0.539 log points for recent immigrants (= +71\%), and +0.233 log points for immigrants without children (= +26\%). These estimates imply that ending enforcement agreements is associated with substantial increases in the size of targeted immigrant subpopulations—driven mainly by recent, single, low-English-proficiency immigrants. In contrast, the estimated effects for Mexican-born immigrants (+0.078 log points = +8\%) and non-Hispanic immigrants (–0.145 log points = –13.5\%) are small and imprecise, suggesting the overall response is driven by Hispanic non-Mexican immigrants, consistent with the long-run decline in the Mexican share of U.S. immigrants \citep{mpi2024}.

Panel B repeats the exercise for a spillover group: foreign-born Hispanic U.S. citizens with similar demographic characteristics. Although not directly targeted for deportation, this group may still experience racial profiling or include recently naturalized former members of the target population. The effects of gaining 287(g) vary: the baseline estimate is +0.277 log points (= +32\%) and statistically significant at the 10 percent level, but the direction of effects across subgroups differs. Mexican-origin spillovers (Column 8) show a +1.208 log-point increase (= +234\%), while recent immigrant spillovers (Column 10) exhibit a –1.128 log-point decrease (= –68\%). These heterogeneous responses suggest that some spillover groups avoid newly activated enforcement areas, whereas others—particularly well-established Mexican-origin spillover communities—grow due to demographic or residential sorting. Losing an agreement yields mostly small, statistically insignificant effects on spillover groups; the only notable estimate is a modest decline for non-Hispanic spillovers (–0.210 log points = -19\%), indicating that positive population responses to 287(g) termination are concentrated within the targeted immigrant groups rather than among Hispanic citizens more broadly.
%these results are kind of all over the map. Some of the populations are fairly small which may explain some of the instability. Any reason why these should show up mostly on the gain variable, while targeted group more shows up on the lose variable? I'm not sure we learn much from the spillover population. Maybe panel B should be the placebo population -- to show that you don't find effects in these subpopulations in the placebo?

\section{Pending discussion}


\bibliography{bib_287}


\section{Tables and figures}    
\begin{spacing}{1}
    

%%%%%% TABLES

%% table 1
\begin{table}[H]
\centering
\caption{ICE Removals from a 287(g) apprehension}
\label{tab:removals}
\adjustbox{ width=0.7\linewidth}{
    \begin{tabular}{rrrll}
\toprule
\toprule
\multicolumn{1}{l}{Characteristics}                         & Removals & Share  &  &  \\\midrule 
\multicolumn{1}{l}{Gender}                                  &          &        &  &  \\Male  & 87,923   & 0.9587 &  &  \\Female & 3,790    & 0.0413 &  &  \\\multicolumn{1}{l}{Age group}                               &          &        &  &  \\0-17  & 82       & 0.0009 &  &  \\18-24 & 17,911   & 0.1953 &  &  \\25-29 & 21,588   & 0.2354 &  &  \\30-34 & 19,568   & 0.2134 &  &  \\35-39 & 14,631   & 0.1595 &  &  \\40-44 & 9,040    & 0.0986 &  &  \\45-49 & 4,985    & 0.0544 &  &  \\50-54 & 2,384    & 0.0260 &  &  \\55-59 & 1,004    & 0.0109 &  &  \\60-64 & 365      & 0.0040 &  &  \\65-69 & 115      & 0.0013 &  &  \\70-74 & 32       & 0.0003 &  &  \\75+   & 9        & 0.0001 &  &  \\\multicolumn{1}{l}{Latin American or} & 90,235   & 0.9839 &  &  \\\multicolumn{1}{l}{Caribbean Citizenship} &   &  &  &  \\\multicolumn{1}{l}{Citizenship}                             &          &        &  &  \\Mexico & 68,038   & 0.7418 &  &  \\Guatemala & 8,042    & 0.0877 &  &  \\Honduras & 6,618    & 0.0722 &  &  \\El Salvador & 5,300    & 0.0578 &  &  \\Nicaragua & 329      & 0.0036 &  &  \\Brazil & 306      & 0.0033 &  &  \\\multicolumn{1}{l}{Seriousness Level of Conviction}         &          &        &  &  \\No Conviction & 16,270   & 0.1774 &  &  \\Level 1 Crime & 25,620   & 0.2793 &  &  \\Level 2 Crime & 9,065    & 0.0988 &  &  \\Level 3 Crime & 40,759   & 0.4444 &  & \bottomrule
\bottomrule
\end{tabular}
    
}

\justifying

\begin{spacing}{1}
\begin{footnotesize}
\noindent \textit{Notes:} This table summarizes characteristics of foreign citizens removed by ICE following a 287(g) apprehension from 2013-2019. Latin American citizenship is identified through the top 50 reported countries. Citizenship reports only the leading 5 countries of citizenship. Source: TRAC Immigration Tools \citep{trac24}.
\end{footnotesize}
\end{spacing}
\end{table}


%% table 2
\newpage
\begin{table}[H]
\begin{center}

\caption{Characteristics of target population in treated and untreated migpumas}
\label{tab:balance}
\adjustbox{ width=0.8\linewidth}{
    \begin{tabular}{lccc}
\toprule
\toprule
 & \multicolumn{3}{c}{Target population} & \multicolumn{3}{c}{Migpuma}  \\
 & Treated & Untreated & Difference   \\
 & (1) & (2) & (3)  \\
\midrule 
 \textbf{Mobility} & & &   \\
 Moved migpuma  & 0.12  & 0.13  & -0.00 \\
 & (0.01) & (0.00) & (0.01)\\
 Moved state  & 0.10  & 0.09  & 0.01 \\
 & (0.00) & (0.00) & (0.01)\\
 Moved from abroad  & 0.09  & 0.07  & 0.02*** \\
 & (0.00) & (0.00) & (0.01)\\
 \textbf{Demographics and education} & & &   \\
 Age  & 26.31  & 26.14  & 0.17 \\
 & (0.09) & (0.08) & (0.12)\\
 Number of children  & 0.21  & 0.21  & -0.00 \\
 & (0.01) & (0.01) & (0.01)\\
 Race: White  & 0.64  & 0.53  & 0.10*** \\
 & (0.01) & (0.01) & (0.01)\\
 Race: Black  & 0.01  & 0.02  & -0.01*** \\
 & (0.00) & (0.00) & (0.00)\\
 High School  & 0.47  & 0.45  & 0.01 \\
 & (0.01) & (0.01) & (0.01)\\
 Poor English  & 0.69  & 0.67  & 0.02* \\
 & (0.01) & (0.01) & (0.01)\\
 In School  & 0.09  & 0.08  & 0.01 \\
 & (0.00) & (0.00) & (0.01)\\
 \textbf{Employment and housing} & & &   \\
 Employed  & 0.84  & 0.84  & 0.01 \\
 & (0.01) & (0.00) & (0.01)\\
 Weeks worked  & 40.01  & 41.39  & -1.37 \\
 & (0.78) & (0.65) & (1.02)\\
 Usual weekly hours worked  & 35.19  & 35.30  & -0.11 \\
 & (0.24) & (0.20) & (0.31)\\
 Wage income  & 21,500.92  & 21,852.45  & -351.53 \\
 & (251.16) & (237.65) & (345.80)\\
 Owns a home  & 0.16  & 0.15  & 0.01 \\
 & (0.01) & (0.00) & (0.01)\\
 Rent price  & 1,008.73  & 1,035.85  & -27.11** \\
 & (7.41) & (8.04) & (10.93)\\
 Mortgage price  & 872.33  & 910.53  & -38.20 \\
 & (31.73) & (27.31) & (41.88)\\
Sample size & 5,963 & 8,151 & \\
\bottomrule
\bottomrule
\\
\end{tabular}
}
\end{center}

\justifying

\begin{spacing}{1}
\begin{footnotesize}
\noindent \textit{Notes:} This table summarizes main characteristics the population targeted by 287(g): males, ages 18-39, born in a Latin American country, not a U.S. citizen, that have spent 10 years or less in the country, that have a high school degree or less, and not currently married. Column 1 summarizes average characteristics for those in migpumas ever treated from 2013-2019 and Column 2 for those in never treated migpumas. Column 3 summarizes the average of those in treated areas minus the average of those in untreated areas. p $<$ 0.01 ***, p $<$ 0.05 **, p $<$0.1 *. 
\end{footnotesize}
\end{spacing}
\end{table}



%% table 4
\newpage
\begin{table}[H]
\centering
\caption{Net effect of an active 287(g) on log total targeted and log total placebo populations}
\label{tab:regpop}
\adjustbox{ width=0.9\linewidth}{
    \begin{tabular}{lcccc}
\toprule
\toprule
 & \multicolumn{2}{c}{Target population} & \multicolumn{2}{c}{Placebo population}  \\
 Log population & (1) & (2)  & (3) & (4)  \\
\midrule 
 Treated migpuma & -0.1357** & -0.1603** & -0.0170 & -0.0456 \\
 & (0.0634) & (0.0650) & (0.0567) & (0.0528) \\
\\
 Controls &  & X &  & X \\
 \textit{R2} & 0.9219 & 0.9236 & 0.9498 & 0.9515  \\
 Untreated pop size & 1,013 & 1,013 & 7,319 & 7,319  \\
Sample Size & 1,505 & 1,505 & 1,505 & 1,505  \\
\bottomrule
\bottomrule
\\
\end{tabular}
}

\justifying
\begin{spacing}{1}
\begin{footnotesize}


\noindent \textit{Notes:} This table summarizes the results from the regression design detailed in Equation \ref{eq:regpop}. Columns 1-2 evaluates the effect of having an active 287(g) agreement on the log total targeted population and Columns 3-4 on the log total placebo population, for comparison. Columns 2 and 3 incorporate controls at the migpuma-year level: log totals of population in different age distributions, race categories, high school attainment, in school status, and that are homeowners. The unit of observation is the migpuma-year level. All specifications include migpuma and year fixed effects and robust standard errors. p $<$ 0.01 ***, p $<$ 0.05 **, p $<$0.1 *. 

\end{footnotesize}
\end{spacing}
\end{table}





\newpage
\begin{table}[H]
\centering
\caption{Effect of gaining and losing a 287(g) agreement on the log total targeted and the log total placebo populations}
\label{tab:didpop}
\adjustbox{ width=0.9\linewidth}{
    \begin{tabular}{lcccc}
\toprule
\toprule
 & \multicolumn{2}{c}{Target population} & \multicolumn{2}{c}{Placebo population}  \\
Log population & (1) & (2)  & (3) & (4) \\
\midrule 
 Gain treatment & -0.2050 & -0.2284* & 0.0518 & 0.0787 \\
 & (0.1249) & (0.1285) & (0.1021) & (0.1012) \\
 Lose treatment & 0.0696 & 0.0583 & -0.1340 & -0.1354 \\
 & (0.0617) & (0.0602) & (0.1538) & (0.1509) \\
\\
 Controls &  & X &  & X \\
 \textit{R2} & 0.5949 & 0.6038 & 0.7401 & 0.7455  \\
 Untreated pop size & 832 & 832 & 1,483 & 1,483  \\
Sample Size & 3,487 & 3,487 & 3,487 & 3,487  \\
\bottomrule
\bottomrule
\\
\end{tabular}
}

\justifying
\begin{spacing}{1}
\begin{footnotesize}


\noindent \textit{Notes:} This table summarizes the results from the difference-in-difference detailed in Equation \ref{eq:joindid}. Columns 1-2 evaluates the effect of gaining a 287(g) agreement and losing a 287(g) agreement on the log total targeted population and Columns 3-4 on the log total placebo population, for comparison. Columns 2 and 3 incorporate controls at the migpuma-year level. The unit of observation is the migpuma-year level. All specifications include migpuma and year fixed effects and robust standard errors. p $<$ 0.01 ***, p $<$ 0.05 **, p $<$0.1 *. 

\end{footnotesize}
\end{spacing}
\end{table}




\begin{landscape}
\begin{table}[t]
\centering
\caption{Effect of gaining and losing a 287(g) agreement on the log total of sectors of the targeted population and the log total of sectors of the placebo}
\label{tab:didhet}
\adjustbox{ width=0.8\linewidth}{
    \begin{tabular}{lcccc}
\toprule
\toprule
 & \multicolumn{7}{c}{Panel A: Target population} \\
\midrule
 & Baseline & Mexican & Poor English & New Immigrant & No children & Non-Hispanic \\
Log population & (1) & (2)  & (3) & (4) & (5) & (6) \\
\midrule 
 Gain treatment & -0.0599 & -0.0389 & -0.0462 & 0.2998 &  &  \\
 & (0.0725) & (0.7040) & (0.1446) & (0.2636) & () & () \\
 Lose treatment & 0.1759*** & 0.0779 & 0.2997*** & 0.5388*** &  &  \\
 & (0.0629) & (0.2199) & (0.0896) & (0.1864) & () & () \\
\\
 Controls &  & X &  & X \\
 \textit{R2} & 0.9333 & 0.6921 & 0.7742 & 0.7178 &  &     \\
 Untreated mean & 7.8908 & 6.0503 & 7.2443 & 5.7511 &  &   \\
Sample Size & 2,115 & 2,115 & 2,115 & 2,115 &  &   \\
\midrule
\midrule
 & \multicolumn{7}{c}{Panel B: Spillover population} \\
 & Baseline & Mexican & Poor English & New Immigrant & No children & Non-Hispanic \\
Log population & (1) & (2)  & (3) & (4) & (5) & (6) \\
 Gain treatment & 0.2772* & 1.2077*** & -0.2509 & -1.1275** & 0.2360 & -0.1788 \\
 & (0.1561) & (0.4614) & (0.3505) & (0.5230) & (0.1605) & (0.1978) \\
 Lose treatment & -0.0472 & 0.2033 & -0.0721 & 0.6577 & 0.0299 & -0.2100* \\
 & (0.1103) & (0.2024) & (0.2591) & (0.5670) & (0.1125) & (0.1195) \\
\\
 Controls & X & X & X & X & X & X \\
 \textit{R2} & 0.8446 & 0.7625 & 0.7505 & 0.6310 &  &     \\
 Untreated mean & 6.4636 & 4.6037 & 3.8017 & 1.8135 &  &   \\
Sample Size & 2,115 & 2,115 & 2,115 & 2,115 &  &   \\
\bottomrule
\bottomrule
\\
\end{tabular}
}

\justifying
\begin{spacing}{1}
\begin{footnotesize}

\noindent \textit{Notes:} This table summarizes the results from the difference-in-difference detailed in Equation \ref{eq:joindid} on sizes of various population sectors. Panel A evaluates the effect of gaining and losing a 287(g) agreement on the size of different and similar sectors of the target population. Columns 1 represents the baseline, corresponding to Column 2 from Table \ref{tab:didpop}. Column 2 restricts the evaluated population to those from Mexican origin, Column 3 to those with poor English level, Column 4 to those with up to two years in the country, Column 5 to those with no children, and Column 6 evaluates non-Hispanic immigrants that share all other characteristics with the target population. Panel B repeats the top panel for the placebo population, obviating effects on new immigrants. p $<$ 0.01 ***, p $<$ 0.05 **, p $<$0.1 *. 
\end{footnotesize}
\end{spacing}
\end{table}
\end{landscape}



\begin{landscape}
\begin{table}[t]
\centering
\caption{Effect of gaining and losing a 287(g) agreement on the log total spillover population}
\label{tab:didhet}
\adjustbox{ width=0.8\linewidth}{
    \begin{tabular}{lcccccc}
\toprule
\toprule
 & Baseline & Mexican & Poor English & New Immigrant & No children & Non-Hispanic \\
Log population & (1) & (2)  & (3) & (4) & (5) & (6) \\
\midrule 
 Gain treatment & 0.3956 & 1.2361 & 0.2350 & -0.1211 & 0.3995 & -0.3974 \\
 & (0.4084) & (0.8413) & (0.5338) & (0.7543) & (0.4358) & (0.3113) \\
 Lose treatment & 0.1232 & -0.1345 & 0.0263 & -1.0161* & 0.1413 & -0.2865 \\
 & (0.2175) & (0.3251) & (0.4745) & (0.6101) & (0.2184) & (0.2281) \\
\\
 Controls & X & X & X & . & X & X \\
 \textit{R2} & 0.8489 & 0.7381 & 0.7104 & . & 0.8361 & 0.8514    \\
 Untreated pop size & 433 & 204 & 70 & . & 359 & 407  \\
Sample Size & 1,505 & 1,505 & 1,505 & . & 1,505 & 1,505  \\
\bottomrule
\bottomrule
\\
\end{tabular}
}

\justifying
\begin{spacing}{1}
\begin{footnotesize}

\noindent \textit{Notes:} This table summarizes the results from the difference-in-difference detailed in Equation \ref{eq:joindid} on sizes of spillover population and various subsectors and a similar sector. The spillover population includes: foreign-born U.S. citizens from Hispanic descent that have naturalized before 2013, with similar non-immigration characteristics to the targeted (males, ages 18-39, with a high school degree or less, not currently married). Columns 1 represents the baseline, mirroring Column 2 from Table \ref{tab:didpop}. Column 2 restricts the evaluated population to those from Mexican origin, Column 3 to those with poor English level, Column 4 to those with up to two years in the country, Column 5 to those with no children, and Column 6 evaluates non-Hispanic immigrants that share all other characteristics with the target population.  p $<$ 0.01 ***, p $<$ 0.05 **, p $<$0.1 *. 
\end{footnotesize}
\end{spacing}
\end{table}
\end{landscape}

\end{spacing}


%%%%% FIGURES


\begin{figure}[H]
\centering
\caption{Yearly trends in 287(g) removals}
\label{fig:agreement_count}
\begin{center}
\adjustbox{width=\linewidth}{
    \includegraphics[width=\linewidth]{output/deportations.png}
}
\end{center}
\justifying
\begin{spacing}{1}
\begin{footnotesize}
\noindent \textit{Notes:} This graph summarizes yearly trends in ICE reomvals following a 287(g) apprehension from 2013-2019. Source: TRAC Immigration Tools \citep{trac24}.
\end{footnotesize}
\end{spacing}
\end{figure}



\begin{figure}[H]
\centering
\caption{Total target population}
\label{fig:agreement_count}
\begin{center}
(a) Total target population
\adjustbox{width=\linewidth}{
    \includegraphics[width=0.7\linewidth]{output/final/total_target2_pop.png}
}
(a) Mean migpuma target population
\adjustbox{width=\linewidth}{
    \includegraphics[width=0.7\linewidth]{output/final/mean_target2_pop.png}
}
\end{center}
\justifying
\begin{spacing}{1}
\begin{footnotesize}
\noindent \textit{Notes:} This graph summarizes yearly trends in ICE reomvals following a 287(g) apprehension from 2013-2019. Source: TRAC Immigration Tools \citep{trac24}.
\end{footnotesize}
\end{spacing}
\end{figure}


%% figure 1

%% figure 2 
\begin{landscape}
\begin{figure}[t]
\centering
\caption{Event study of the effect of losing and gaining a 287(g) agreement on the log total targeted and placebo populations}
\label{fig:didpop}

\begin{minipage}{0.44\linewidth}
\centering
\adjustbox{width=\linewidth}{\includegraphics{output/final/logtargetpop_gain_estudy.png}}
\end{minipage}
\begin{minipage}{0.44\linewidth}
\centering
\adjustbox{width=\linewidth}{\includegraphics{output/final/logtargetpop_lost_estudy.png}}
\end{minipage}
\begin{minipage}{0.44\linewidth}
\centering
\adjustbox{width=\linewidth}{\includegraphics{output/final/logplacebopop_gain_estudy.png}}
\end{minipage}
\begin{minipage}{0.44\linewidth}
\centering
\adjustbox{width=\linewidth}{\includegraphics{output/final/logplacebopop_lost_estudy.png}}
\end{minipage}

\justifying
\begin{spacing}{1}
\begin{footnotesize}
\noindent \textit{Notes}: These event studies explore the effect of gaining and losing an agreement of the log of total population, using the specifications of Table \ref{tab:didpop}. Panel (a) plots the effect of gaining an agreement on the log of total target population and Panel (b), the effect of losing an agreement. Panel (c) plots the effect of gaining an agreement on the log of total placebo population and Panel (d), the effect of losing an agreement. All specifications include migpuma-year controls (log totals of population in different age distributions, race categories, high school attainment, in school status, and that are homeowners), migpuma and year fixed effects, and robust standard errors.
\end{footnotesize}
\end{spacing}

\end{figure}

\end{landscape}




\renewcommand{\thefigure}{A\arabic{figure}}
\renewcommand{\thetable}{A\arabic{table}}
\setcounter{figure}{0}
\setcounter{table}{0}
\setcounter{page}{1}

\section*{Appendix: Applying propensity score}
To address selection of migpumas into the 287(g) program, I weight migpumas using propensity scores. These are calculated through a logit estimation using 2013 (start of period of observation) migpuma characteristics and weighting using the total population of target immigrants to estimate the probability of having an active 287(g) during this period. 

%% table 1
\begin{figure}[h!]
\centering
\caption{Density of propensity score}
\label{fig:prop_score}
\begin{center}
\adjustbox{width=0.7\linewidth}{
    \includegraphics[width=\linewidth]{output/final/prop_score.png}
}
\end{center}
\justifying
\begin{spacing}{1}
\begin{footnotesize}
\noindent \textit{Notes:} This graph shows the density of the probability of a migpuma been treated at any point from 2013-2019 for the treatment group, the control, and the control upon weighting on propensity scores.
\end{footnotesize}
\end{spacing}
\end{figure}

The characteristics include 2013 total adult population and, to match on composition of localities, using 2013 shares of
\begin{spacing}{1}
\begin{enumerate}
    \item Target population
    \item Foreign born
    \item Young (18-39)
    \item Marital status
    \item Red state
    \item Texas
    \item State treatment
    \item Race
    \item Ethnicity
    \item Education
    \item English level
\end{enumerate}
\end{spacing}

Though all other specification exclude migpumas that always have an active 287(g) during 2013-2019, as they are ultimately absorbed by migpuma fixed effects, I include them for finding propensity scores as they provide information of the type of localities with a 287(g) agreement. 

\begin{table}[h!]
\centering
\caption{Effect of gaining and losing a 287(g) agreement on the log total targeted and the log total placebo populations with propensity score weighting}
\label{tab:didpopprop}
\adjustbox{ width=0.7\linewidth}{
    \begin{tabular}{lcccc}
\toprule
\toprule
 & \multicolumn{2}{c}{Target population} & \multicolumn{2}{c}{Placebo population}  \\
Log population & (1) & (2)  & (3) & (4) \\
\midrule 
 Gain treatment & 0.0145 & 0.0005 & 0.1877 & 0.1550 \\
 & (0.1681) & (0.1680) & (0.1305) & (0.1296) \\
 Lose treatment & -0.0095 & 0.0023 & -0.1278*** & -0.1155** \\
 & (0.0722) & (0.0718) & (0.0457) & (0.0451) \\
\\
 Controls &  & X &  & X \\
 \textit{R2} & 0.4980 & 0.5025 & 0.7589 & 0.7648  \\
 Untreated mean & 6.2827 & 6.2827 & 7.0167 & 7.0167  \\
Sample Size & 7,905 & 7,905 & 7,905 & 7,905  \\
\bottomrule
\bottomrule
\\
\end{tabular}
}

\justifying
\begin{spacing}{1}
\begin{footnotesize}


\noindent \textit{Notes:} This table summarizes the results from the difference-in-difference detailed in Equation \ref{eq:joindid}, when including propensity scores weights. Columns 1-2 evaluates the effect of gaining a 287(g) agreement and losing a 287(g) agreement on the log total targeted population and Columns 3-4 on the log total placebo population, for comparison. Columns 2 and 3 incorporate controls at the migpuma-year level. The unit of observation is the migpuma-year level. All specifications include migpuma and year fixed effects and robust standard errors. p $<$ 0.01 ***, p $<$ 0.05 **, p $<$0.1 *. 

\end{footnotesize}
\end{spacing}
\end{table}


I apply these propensity scores to our difference-in-difference specification (Equation \ref{eq:joindid}) in Table \ref{eq:didpopprop}. For the target population, we see that the propensity weighting, increases the the effect of gaining treatment, though these still have large standard errors, and dampens the effect of losing treatment. The effect of losing treatment is much lower at +9.5\% in the setup obviating migpuma-year control. Upon accounting for migpuma-year controls, the effect of losing an agreement returns to a closer level to Column 2 from Table \ref{tab:didpop}, at +0.1374 log points or +14.7\%, significant at the 5 percent level. The estimates of losing treatment for the placebo group are much larger in magnitude though the standard errors are also quite large.






\end{document}
