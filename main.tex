
\documentclass{article}
\usepackage{ragged2e} 
\usepackage{graphicx} % Required for inserting images
% Document dependencies
\usepackage[utf8]{inputenc}
\usepackage[square,sort,comma,numbers]{natbib}
\bibliographystyle{agsm}
\bibpunct{(}{)}{;}{a}{,}{,}
\usepackage{graphicx, adjustbox, rotating}
\usepackage{booktabs, url}
\usepackage{comment}
\usepackage{float}
\usepackage{pdflscape}
\usepackage{caption, subcaption}
\usepackage[margin=1in]{geometry} %margins
\usepackage[colorlinks = true, urlcolor = blue, linkcolor = blue, citecolor = blue]{hyperref}
\usepackage[parfill]{parskip} % first line skip 
\setlength{\parindent}{15pt} %for paragraph indent
\usepackage{indentfirst} %for first paragraph indent
\usepackage{setspace} %line spacing
 \usepackage{pdfpages}

\DeclareGraphicsExtensions{.pdf,.png,.jpg}
\pdfgentounicode=1

\title{The Effects of 287(g) Agreements on Immigrant's Mobility Decisions}
\author{Mena Kiser}
\date{\today}

\begin{document}
\doublespacing
\maketitle

\begin{abstract}
    287(g) agreements create partnerships between state and local agencies and ICE, allowing for deputized officers to perform some of ICE duties; this program led to deportations of immigrants of Hispanic origin burgeoning between 2011-2014. Out of fear of racial discrimination, Hispanic non-citizens may be more prone to moving out of counties with active 287(g) agreements. I explore this through a staggered difference-in-difference design evaluating the effect of living in a county with an active 287(g) agreement among the population targeted by this policy. As counties can volunteer into this agreement, I address selection concerns by applying a propensity score weights using parameters determinant of a county being exposed to the program at any point during our period of observation.
\end{abstract}

%\section{Introduction}

\section{287(g) Agreements}
The 287(g) program establishes partnerships between state and local enforcement agencies and ICE for local officers to exercise ICE duties to some extent. Agencies can voluntarily request participation in the program and ICE ultimately decides which agencies are included in the program and enters negotiations for an agreement. Agreements are negotiated between DHS and local agencies and supervised by ICE; they establish delegation of authority to a determined number of officers. After an agreement expires, DHS is not obligated to renew it. Not all agreements include a specific expiration date, and once an agreement is entered into, it may be terminated at any time by either party.

This policy was officially enacted in 1996 as part of the Illegal Immigration Reform and Immigrant Responsibility Act (IIRIRA); however, no agencies joined the program until 2002 (Figure \ref{fig:timeline}). In 2008, ICE created a standard template for Memorandum of Agreements (MOA), simplifying the process for LEAs to enroll. Between 2006 and 2009, at least 56 agencies joined the program, partially due to the simplification of the process. In September of 2025, DHS reported over 1,000 current agreements, most of which were entered starting 2019. Changes in recruiting, funding, and ICE and DHS priorities for identifying undocumented immigrants may change the program's intensity.

This program can take three different models (and hybrids of these) based on the needs, capacity, and interest of the agency and ICE, we can continuously observe only one type, jail enforcement modality, for the full extent of our period of observation (2013-2019). Under this type of agreement local officers can interrogate and place detainers (requests to maintain in custody for up to 48 hours) for suspected noncitizens who have been arrested. The task force model allows local officers to interrogate suspected noncitizens encounter in every day activities was rescinded in 2012\footnote{The decision to rescind this model was largely due to concerns of racial discrimination. The rescission entailed task force agreements would not be renewed but we observe some counties had an active agreement up to 2013.} and reinstated in 2025. The warrant service officer model, under which local officers to receive ICE training to execute immigration warrants, was first introduced in May 2019. We define exposure to a 287(g) agreement as being exposed to any model of 287(g), understanding that jail enforcement modality is the predominant model being captured during this period. 

\section{Data}
The main datasets used in this paper come from the ICE websites,  Syracuse University Transactional Records Access Clearinghouse (TRAC) reports, and the American Community Survey (ACS). 

Through the Internet Archive's Wayback Machine, I can extract snapshots from ICE websites listing LEAs with current 287(g) agreements starting. These lists are retrievable starting in 2011. All lists include the name of the agency, the date it was signed, and the type of agreement entered. For each year, I can obtain a minimum of 3 snapshots from different days in different months, as seen in Table \ref{tab:retrievals}. Having access to multiple observations within a year allows me to identify the dates when agreements were terminated, as indicated by their absence from subsequent retrievals. I match agencies to their local geographic area by name using a Bureau of Justice Statistics (\citeyear{data:leaxwalk}) crosswalk from local enforcement agency to county. Exposure to the 287(g) program is defined as having any active agreement in the county of residence during the survey year. I distinct between local agreememts and state agreements as they may have different effects and following....


I obtain details on the demographics of those targeted by 287(g) from the Transactional Records Access Clearinghouse (\citeyear{trac24}), an organization founded by the Syracuse University that gathers and distributes immigration statistics obtained from government agencies through the Freedom of Information Act (FOIA). From reports on ICE removals from 2013--2019 initiated with a 287(g) Program apprehension, summarized in Table \ref{tab:removals}, we can see that 76\% of the deportees were of ages 18--39, 98\% had citizenship in a Latin American country, and 96\% were male. We use these characteristics to identify the population targeted by this policy: males, Latin American country of origin, ages 18-39, foreign-born, non-citizen.

Individual characteristics are extracted from the 2013-2019 American Community Survey (ACS), obtained through the Integrated Public Use Microdata Series (IPUMS) \citep{data:acs}. The ACS is a yearly cross-sectional, one percent, annual survey of households in the United States. Through this survey we can identify foreign-born respondents using place of birth and citizenship status; however, this survey does not ask about current legal status or status at entry. I restrict the targeted population to those with a U.S. arrival after 2007 to rule out DACA eligibility.
To hone into a mobile population, I restrict the targeted population to those with low-skill (educational attainment of High School degree or less), those not currently married, those without own children in the household, and recent arrivals (with 10 years or less in the country)\footnote{Restricting to those with high school or less rules out international students pursuing a college degree. Excluding those not currently married rules out those eligible to legal permanent status through their spouse.}. In Table \ref{tab:sumstat}, we can observe characteristics of the targeted population across level of exposure.

The ACS includes an individual's state and county of current residence and previous year residence. The smallest geographic unit in the ACS is PUMA; however

This source also provides  place of work county, commuting time, current occupation, industry, earnings, and more which will help us understand other effects of this policy.


\section{Empirical design}
To identify the effects of this policy, I use a difference-in-difference design interacting being part of the target population and exposure, as shown in Equation \ref{eq:did}. However, as treatment is quite unbalanced, I use a propensity score matching to re-weight untreated counties. Through a logit estimation regressing on a county's share of targeted population, low skill population, foreign population, young population, total population, mean income, house price (either rent or mortgage), and being a predominantly red state. The predicted exposure score along with the original non-zero scores are applied to the untreated. 
As the cross-sectional data from ACS uses probability weights, I multiply these by the propensity score obtained and apply this for all analysis. In Table \ref{tab:sumstat}, we can observe how relevant characteristics for the treated and untreated groups changes when applying the propensity score weights.
\begin{equation}
    \label{eq:did}
    in-migration_{icy} = \beta_{0} + \beta_{1}exposure_{cy}*targeted_{i} +  \beta_{2}exposure_{cy} + \beta_{3}targeted_{i} + \beta_{42}W_{y} + \beta_{53}Z_{c} + \epsilon_{cy} 
\end{equation}

To first test the parallel trends assumption, we can evaluate the effect of exposure on migration decisions, following Equation \ref{eq:reg1}, in which in--migration can take any move (including within county), moving counties, and moving states. I include the results for this estimation in Table \ref{tab:regtp}. We can see that in the targeted population, 287(g) exposure increases mobility within county, across counties, and across states, with a higher significance level at county level. These results become stronger and more significant when applying propensity weights. In the placebo population, we observe mostly negative effects and not-statistically significant effects of 287(g) programs in mobility.
\begin{equation}
    \label{eq:reg1}
    in-migration_{icy} = \beta_{0} + \beta_{1}exposure_{cy} +  \beta_{42}W_{y} + \beta_{53}Z_{c} + \epsilon_{cy} 
\end{equation}

These results suggest there are differential effects in migration for the non-citizen targeted population, distinct from similar citizens. However, the direction of the effect has the opposite of the sign expected. I initially hypothesized that areas in which 287(g) is currently enacted, targeted immigrants would be less likely to move to areas where these agreements have been recently enacted. We are seeing that areas with active MOAs are more likely to have newcomers from the targeted population. A possible explanation for this, is that deporting immigrants in this group creates job vacancies and may potentially increase wages if demand for low-skilled Hispanic workers is not met, thus immigrants move to these areas in pursue of better economic opportunities. To validate this new hypothesis we must also check economic outcomes.

\begin{comment}

in migration: previous year is different from the current year
\begin{equation}
    in-migration_{icy} = \beta_{0} + \beta_{1}exposure_{cy}*targeted_{i} +  \beta_{2}exposure_{cy} + \beta_{3}targeted_{i} + \beta_{42}W_{y} + \beta_{53}Z_{c} + \epsilon_{cy} 
\end{equation}

out migration 
\begin{equation}
    out-migration_{ic'y'} = \beta'_{0} + \beta'_{1}exposure_{c'y'}*targeted_{i} + \beta'_{2}exposure_{c'y'} + \beta'_{3}targeted_{i} + \beta'_{42}W_{y'} + \beta'_{43}Z_{c'} + \epsilon_{c'y'} 
\end{equation}
\end{comment}

\bibliography{bib_287.bib}

\newpage
\section{Tables and Figures}

\begin{table}[h]
\centering
\caption{ICE Removals from a 287(g) apprehension}
\label{tab:removals}
\adjustbox{ width=0.5\linewidth}{
    \begin{tabular}{rrrll}
\toprule
\toprule
\multicolumn{1}{l}{Characteristics}                         & Removals & Share  &  &  \\\midrule 
\multicolumn{1}{l}{Gender}                                  &          &        &  &  \\Male  & 87,923   & 0.9587 &  &  \\Female & 3,790    & 0.0413 &  &  \\\multicolumn{1}{l}{Age group}                               &          &        &  &  \\0-17  & 82       & 0.0009 &  &  \\18-24 & 17,911   & 0.1953 &  &  \\25-29 & 21,588   & 0.2354 &  &  \\30-34 & 19,568   & 0.2134 &  &  \\35-39 & 14,631   & 0.1595 &  &  \\40-44 & 9,040    & 0.0986 &  &  \\45-49 & 4,985    & 0.0544 &  &  \\50-54 & 2,384    & 0.0260 &  &  \\55-59 & 1,004    & 0.0109 &  &  \\60-64 & 365      & 0.0040 &  &  \\65-69 & 115      & 0.0013 &  &  \\70-74 & 32       & 0.0003 &  &  \\75+   & 9        & 0.0001 &  &  \\\multicolumn{1}{l}{Latin American or} & 90,235   & 0.9839 &  &  \\\multicolumn{1}{l}{Caribbean Citizenship} &   &  &  &  \\\multicolumn{1}{l}{Citizenship}                             &          &        &  &  \\Mexico & 68,038   & 0.7418 &  &  \\Guatemala & 8,042    & 0.0877 &  &  \\Honduras & 6,618    & 0.0722 &  &  \\El Salvador & 5,300    & 0.0578 &  &  \\Nicaragua & 329      & 0.0036 &  &  \\Brazil & 306      & 0.0033 &  &  \\\multicolumn{1}{l}{Seriousness Level of Conviction}         &          &        &  &  \\No Conviction & 16,270   & 0.1774 &  &  \\Level 1 Crime & 25,620   & 0.2793 &  &  \\Level 2 Crime & 9,065    & 0.0988 &  &  \\Level 3 Crime & 40,759   & 0.4444 &  & \bottomrule
\bottomrule
\end{tabular}
}
\end{table}

\justifying
\
\begin{spacing}{1}
\begin{footnotesize}
\noindent \textit{Notes:} This table summarizes characteristics of foreign citizens removed by ICE following a 287(g) apprehension from 2013-2019. Latin American citizenship is identified through the top 50 reported countries. Citizenship reports only the leading 5 countries of citizenship. Source: TRAC Immigration Tools \citep{trac24}.
\end{footnotesize}
\end{spacing}

\newpage
\section{Tables and Figures}

\begin{table}[h]
\centering
\caption{Retrievals of active Memorandum of Agreements}
\label{tab:retrievals}
\adjustbox{ width=0.4\linewidth}{
    \begin{tabular}{lccc}
\toprule
\toprule
 & \multicolumn{3}{c}{Distinct retrivals} \\
 Year & Days & Months & LEAs \\
\midrule 
2011 & 3 & 3 & 71 \\
2012 & 37 & 8 & 68 \\
2013 & 6 & 6 & 57 \\
2014 & 3 & 3 & 37 \\
2015 & 16 & 9 & 34 \\
2016 & 13 & 8 & 34 \\
2017 & 125 & 12 & 62 \\
2018 & 306 & 12 & 79 \\
2019 & 332 & 12 & 94 \\
\bottomrule
\bottomrule
\end{tabular}
}
\end{table}

\justifying
\
\begin{spacing}{1}
\begin{footnotesize}
\noindent \textit{Notes:} This table quantifies the number of retrievals from distinctive days, months, and pertaining different Local Enforcement Agencies (LEAs).
\end{footnotesize}
\end{spacing}


\newpage
\begin{landscape}
\begin{table}[h]
\centering
\caption{Summary statistics by exposure and target group}
\label{fig:timeline}
\adjustbox{ width=\linewidth}{
    \begin{tabular}{lccc}
\toprule
\toprule
 & Exposed & \multicolumn{2}{c}{Never Exposed} \\
 &  &  & Propensity weighted \\
 & (1) & (2) & (3)   \\
\midrule 
 Exposure  & 0.42 & 0.00 & 0.00\\
 Any move  & 0.28 & 0.26 & 0.27\\
 Moved migpuma  & 0.12 & 0.13 & 0.13\\
 Moved state  & 0.10 & 0.09 & 0.10\\
 Age  & 26.31 & 26.14 & 26.15\\
 Race: White  & 0.64 & 0.53 & 0.63\\
 Race: Black  & 0.01 & 0.02 & 0.01\\
 Race: Asian  & 0.00 & 0.00 & 0.00\\
 High School  & 0.47 & 0.45 & 0.47\\
 Poor English  & 0.69 & 0.67 & 0.66\\
 In School  & 0.09 & 0.08 & 0.09\\
 Number of children  & 0.21 & 0.21 & 0.21\\
 Employed  & 0.84 & 0.84 & 0.83\\
 Weeks worked  & 40.01 & 41.41 & 42.00\\
 Usual weekly hours worked  & 35.19 & 35.29 & 34.68\\
 Wage income  & 21,500.92 & 21,877.51 & 21,859.27\\
 Owns a home  & 0.16 & 0.15 & 0.19\\
 Rent price  & 1,008.73 & 1,040.50 & 1,046.92\\
 Mortgage price  & 872.33 & 916.56 & 820.27\\
 Sample size  & 2,509.00 & 11,489.00 & 11,489.00\\
\bottomrule
\bottomrule
\end{tabular}
}
\end{table}

\justifying
\begin{spacing}{1}
\begin{footnotesize}
\noindent \textit{Notes:} This table summarizes main characteristics by exposure and population type. Columns 1-2 and 5-6 restrict the sample to immigrants likely targeted by 287(g) agreements: foreign-born non-citizen, Hispanic, male, of ages 18-39, with High School degree or less, that immigrated after 2007, have been in the US for 10 years or less, and are not currently married. Columns 3-4 and 7-8 restricts the sample to comparable individuals, meeting all criteria from the targeted sample except they must be US-born citizens (thus no immigration time requirements). Odd-numbered columns restrict the sample to (untreated) zero exposure counties and even-numbered columns restrict the sample to non-zero exposure counties. Columns 1-4 apply probability weights as extracted from ACS, while columns 5-8 apply weights obtained from propensity score matching.
\end{footnotesize}
\end{spacing}
\end{landscape}




\newpage
\begin{table}[h]
\centering
\caption{Exposure to 287(g)  on targeted and placebo populations}
\label{tab:regtp}
\adjustbox{ width=0.7\linewidth}{
    \begin{tabular}{lcccc}
\toprule
\toprule
 \multicolumn{5}{c}{Panel A: In migration}  \\
 & & & \multicolumn{2}{c}{Propensity weighting}  \\
 & Targeted & Placebo & Targeted & Placebo \\
 & (1) & (2)  & (3) & (4)  \\
\midrule 
 Any move & 0.0429*** & 0.0074 & 0.0377*** & 0.0056 \\
 \textit{SE} & (0.0142) & (0.0048) & (0.0143) & (0.0044) \\
 \textit{R2} & 0.1137 & 0.0890 & 0.0773 & 0.0709  \\
\\
 Move migpuma & 0.0185 & 0.0014 & 0.0123 & -0.0008 \\
 \textit{SE} & (0.0121) & (0.0026) & (0.0118) & (0.0020) \\
 \textit{R2} & 0.1240 & 0.0541 & 0.0747 & 0.0317  \\
\\
 Move state & 0.0208* & 0.0014 & 0.0143 & 0.0004 \\
 \textit{SE} & (0.0117) & (0.0015) & (0.0116) & (0.0013) \\
 \textit{R2} & 0.1285 & 0.0494 & 0.0754 & 0.0214  \\
\\
Sample Size  & 16,025  & 124,229  & 15,903  & 123,159 \\
\midrule
\midrule
 \multicolumn{5}{c}{Panel B: Out migration}  \\
 & & & \multicolumn{2}{c}{Propensity weighting}  \\
 & Targeted & Placebo & Targeted & Placebo \\
 & (5) & (6)  & (7) & (8)  \\
\midrule 
 Any move & 0.0230 & 0.0037 & 0.0362 & 0.0017 \\
 \textit{SE} & (0.0209) & (0.0089) & (0.0224) & (0.0076) \\
 \textit{R2} & 0.3438 & 0.1101 & 0.3326 & 0.1134  \\
\\
 Move migpuma & -0.0010 & 0.0030 & -0.0035 & -0.0002 \\
 \textit{SE} & (0.0139) & (0.0077) & (0.0104) & (0.0058) \\
 \textit{R2} & 0.6828 & 0.1475 & 0.7530 & 0.2531  \\
\\
 Move state & 0.0049 & -0.0030 & 0.0048 & -0.0022 \\
 \textit{SE} & (0.0059) & (0.0034) & (0.0044) & (0.0027) \\
 \textit{R2} & 0.8304 & 0.2618 & 0.8940 & 0.3906  \\
\\
Sample Size  & 14,851  & 111,680  & 14,715  & 110,596 \\
\bottomrule
\bottomrule
\end{tabular}
}
\end{table}

\justifying
\begin{spacing}{1}
\begin{footnotesize}
\noindent \textit{Notes:} This table compares the effect of exposure to a 287(g) program across targeted population and a placebo population defined by those who meet all criteria of targeted population except they are US-born citizens. Columns 3-4 adjust the ACS weights by the propensity score. Each cell represents a different regression. Individual controls are included in all regressions: age, race, educational attainment, english language states, having health insurance, being in school, and owning a home. Standard errors are set at the county-year level and county and year fixed effects are absorbed. p $<$ 0.01 ***, p $<$ 0.05 **, p $<$0.1 *. 

\end{footnotesize}
\end{spacing}



\newpage
\begin{landscape}
    
\begin{figure}[h]
\centering
\caption{Timeline of the evolution of 287(g) agreements}
\label{tab:regtp}
\adjustbox{ width=\linewidth}{
    \includegraphics{output/timeline.png}
}
\end{figure}

\justifying
\begin{spacing}{1}
\begin{footnotesize}
\noindent \textit{Notes: This timeline includes events relevant to the current panorama of the 287(g) program.}

\end{footnotesize}
\end{spacing}

\end{landscape}

\end{document}
