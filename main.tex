
\documentclass{article}
\usepackage{ragged2e} 
\usepackage{graphicx} % Required for inserting images
% Document dependencies
\usepackage[utf8]{inputenc}
\usepackage[square,sort,comma,numbers]{natbib}
\bibliographystyle{agsm}
\usepackage{caption, subcaption}
\usepackage{amsmath}
\bibpunct{(}{)}{;}{a}{,}{,}
\usepackage{graphicx, adjustbox, rotating}
\usepackage{booktabs, url}
\usepackage{comment}
\usepackage{float}
\usepackage{pdflscape}
\usepackage{caption, subcaption}
\usepackage[margin=1in]{geometry} %margins
\usepackage[colorlinks = true, urlcolor = blue, linkcolor = blue, citecolor = blue]{hyperref}
\usepackage[parfill]{parskip} % first line skip 
\setlength{\parindent}{15pt} %for paragraph indent
\usepackage{indentfirst} %for first paragraph indent
\usepackage{setspace} %line spacing
 \usepackage{pdfpages}

\DeclareGraphicsExtensions{.pdf,.png,.jpg}
\pdfgentounicode=1

\title{The Effects of 287(g) Agreements on Immigrant's Mobility Decisions}
\author{Mena Kiser}
\date{\today}

\begin{document}
\doublespacing
\maketitle

\begin{abstract}
    287(g) agreements create partnerships between state and local agencies and ICE, allowing for deputized officers to perform some of ICE duties; this program led to deportations of immigrants of Hispanic origin burgeoning between 2011-2014. Out of fear of racial discrimination, Hispanic non-citizens may be more prone to moving out of counties with active 287(g) agreements. I explore this through a staggered difference-in-difference design evaluating the effect of living in a county with an active 287(g) agreement among the population targeted by this policy. As counties can volunteer into this agreement, I address selection concerns by applying a propensity score weights using parameters determinant of a county being exposed to the program at any point during our period of observation.
\end{abstract}

\section{Introduction}

% 287g information
The 287(g) program establishes partnerships between state and local enforcement agencies and ICE for local officers to exercise ICE duties. Agencies can voluntarily request participation in the program and ICE ultimately decides which agencies are included in the program and enters negotiations for an agreement. Agreements are negotiated between DHS and local agencies and supervised by ICE, establishing delegation of authority to a determined number of officers. After an agreement expires, DHS is not obligated to renew it. Not all agreements include a specific expiration date, and once an agreement is entered into, it may be terminated at any time by either party.

This policy was officially enacted in 1996 as part of the Illegal Immigration Reform and Immigrant Responsibility Act (IIRIRA); however, no agencies joined the program until 2002 (SOURCE XX). The number of participating agencies varies depending on changes to standard templates of agreements and differences in recruiting and funding. Between 2006 and 2009, at least 56 agencies joined the program, partially due to the simplification of the contract negotiating process. In September of 2025, DHS reported over 1,000 current agreements, most of which were entered starting 2019. Changes in recruiting, funding, and ICE and DHS priorities for identifying undocumented immigrants may change the program's intensity.

This program can take three different models (and hybrids of these) based on the needs, capacity, and interest of the agency and ICE, we can continuously observe only one type, jail enforcement modality, for the full extent of our period of observation (2012-2019). Under this type of agreement local officers can interrogate and place detainers (requests to maintain in custody for up to 48 hours) for suspected noncitizens who have been arrested. The task force model allows local officers to interrogate suspected noncitizens encounter in every day activities was rescinded in 2012\footnote{The decision to rescind this model was largely due to concerns of racial discrimination. The rescission entailed task force agreements would not be renewed but we observe some counties had an active agreement up to 2013.} and reinstated in 2025. The warrant service officer model, under which local officers to receive ICE training to execute immigration warrants, was first introduced in May 2019. We define our treatment as having an active 287(g) agreement in an individual locality as being exposed to any model of 287(g), understanding that jail enforcement modality is the predominant model being captured during this period. 

% branch 1: research on the effect of 287g

% branch 2: research on the effect of reducing or increasing the risk of deportation

% branch 3: research specific to mobility

This study is novel as it uncovers the effects of 287(g) over a span of 8 years at a national level and it focuses on mobility outcomes that could explain the labor market mechanism seen in the literature. 

\section{Data}
The main datasets used in this paper come from the ICE websites and the American Community Survey (ACS). 

Obtaining a yearly list of active 287(g) agreements has posed an issue to other researchers (e.g. XX). Through the Internet Archive's Wayback Machine (XX), I can extract snapshots from ICE websites listing agencies with current 287(g) agreements, retrievable starting 2011. All lists include the name of the agency, the date it was signed, and the type of agreement entered. I then match this to the only ICE list listing signed 287(g) covering multiple years (2012, 2013, 2019) and find a 100\% match in agency names (CITATION XXX). I match agencies to their local geographic area using a Bureau of Justice Statistics (\citeyear{data:leaxwalk}) crosswalk and then use a GEOCORR crosswalk to obtain a match to a PUMA (2010 version). For each year, I can observe if a given locality had an active agreement. Following XXX CITATION, I distinct between local agreements and state agreements (i.e. state-level departments of corrections) as they have different enforcement mechanisms and enforcement to outcomes of interest.

I obtain details on the demographics of those targeted by 287(g) from the Transactional Records Access Clearinghouse (\citeyear{trac24}), an organization distributing statistics obtained from government agencies through the Freedom of Information Act (FOIA). From reports on ICE removals from 2012--2019 initiated with a 287(g) Program apprehension, summarized in Table \ref{tab:removals}, we can see that 76\% of the deportees were of ages 18--39, 98\% had citizenship in a Latin American country, and 96\% were male. We use these characteristics to identify the population targeted by this policy: males, Latin American country of origin, ages 18-39, foreign-born, non-citizen. We can also observe yearly trends in removals, as illustrated in Figure XXX, and we can see that during our period of observation 2012 removals reached close to 20,000. Though this number is small relative to the total population of targeted immigrants (XX\%, using 2012 totals from our ACS sample), research has shown how the fear induced by this policy is salient among Hispanic communities, and leads to changes in labor market decisions such as XX.

Individual characteristics are extracted from the 2012-2019 American Community Survey (ACS), obtained through the Integrated Public Use Microdata Series (IPUMS) \citep{data:acs}. The ACS is a yearly cross-sectional, one percent, annual survey of households in the United States. Through this survey we can identify foreign-born respondents using place of birth and citizenship status; however, this survey does not ask about current legal status or status at entry. I specify the targeted population using the predominant characteristics of removed immigrants (males, Latin American country of origin, ages 18-39, foreign-born, non-citizen) and further restrict this sample to those with a U.S. arrival after 2007 to rule out DACA eligibility. As we focus in migration patterns we focus on individuals mobile individuals: not currently married, also ruling out citizenship eligibility through a spouse, and those with less than ten years in the country, and low-skill individual (educational attainment of High School degree or less).  The ACS defines Public Use Microdata Areas (PUMAs), non-overlapping, statistical geographic areas that partition each state or equivalent entity into geographic areas containing no fewer than 100,000 people each--thus this is never missing for any individual. Migration indicators provide information using migration PUMAs (migpumas), a geographic unit that aggregates one or multiple PUMAs. We use migpuma as our geographic unit of analysis as we are able to identify treatment and mobility at this level. 

 In Table \ref{tab:balance}, we can observe characteristics of the targeted population across never treated migpumas and migpumas that were ever treated. The target population in treated and never treated migpumas have statistically insignificant differences in migpuma and state mobility, but treated migpumas are more likely to have movers from abroad by (2 percentage points). Individuals in treated migpumas are similar in demographics and educational level: they are close in age and race composition, and attain high school and are currently in school at similar rates. We observe a small but significant difference in English level, signaling those in treated migpumas are more likely to have low English level. They are also similar in employment covariates: employed at similar rates, similar total weeks worked, usual hours, and close wage income (though those in treated migpumas have lower income levels). They also have similar home ownership rates and mortgage price, but we can identify a small but significant difference of rent price. Overall, this signals that the targeted population in migpumas with different treatment status are similar in demographics, employment, and housing characteristics.

During our period of observation (2012--2019), we can observe migpumas that gain treatment (an active 287(g) agreement), lose an agreement, that never receive an agreement, or always have an active agreement. A total of 118 migpumas are treated at any point between 2012--2019, of these, 26 are always treated, 73 gain treatment, and 21 lose treatment. We can observe the yearly trends in number of treated migpumas and the geographic distribution of these in Figure \ref{fig:treatdistribution}. 



\section{Empirical design}

Our main sample of observation spans from 2012-2019 and includes migpumas in states with migpumas with any 287(g) active agreement during this period. I focus on a population specially targeted and more responsive to policy changes, as detailed in the previous section, this includes immigrants with the following characteristics: males, ages 18-39, born in a Latin American country, not a U.S. citizen, that have spent 10 years or less in the country, that have a high school degree or less, and not currently married.

To identify the effects of this policy on in-migration, I start with a regression analysis evaluating the effect of having an active 287(g) agreement in any given year on an individual's in migration decisions. I detail this design in Equation  \ref{eq:regpr},  in which the $Y$, outcome variables, is a dummy for having moved in the past year for an individual \textit{i}, currently residing in migpuma \textit{m}, during year \textit{y}. The treatment is defined by having an active agreement in the migpuma of residence during the focus year. I also include individual controls $X_{i}$ (age, race, high school attained, currently in school, poor level of English, and a dummy for owning a house) migpuma fixed effects $W_{m} $ and year fixed effects $Z_{y}$ and cluster standard errors at the migpuma and year level. The coefficient of interest $\beta_{1}$ informs us of the net effect of having an active agreement on the chances of having moved to the migpuma of observation. I focus on in-migration, rather than out-migration because it is possible some of these immigrants are leaving the country alltogether thus we would not be able to identify if they lived in treated migpuma the previous year.
\begin{equation}
    \label{eq:regpr}
    Y_{imy} = \beta_{0} + \beta_{1}\text{active agreement}_{my} + \alpha_{1}X_{i} +  \alpha_{2}W_{m} + \alpha_{3}Z_{y} + \epsilon_{my} 
\end{equation}

As evaluating the probability of having moved to a treated migpuma relies in the overall number and composition (newcomers and stayers) of target immigrants in the locality, it is possible $\beta_{1}$ in Equation  \ref{eq:regpr} captures an outcome based on an evolving sample. Thus, I turn to evaluating the effect of the treatment in the total target population in an area. In this specification, detailed in Equation \ref{eq:regpop}, the outcome is the log target population at the migpuma and year level. The treatment, migpuma fixed effects, and year fixed effects are defined the same as in the previous equation. We add controls at the migpuma-year level, instead of at the individual level, these controls include log totals of population in different age distributions, race categories, high school attainment, in school status, and that are homeowners. These controls mirror those in the previous specification and are included to capture differences across areas with aging population, and changing in racial and education composition, or difficulty of owning a home. The coefficient of interest ($\beta'_{1}$) measures the percent change in the total target population in a migpuma when an active agreement is in place, relative to the same migpuma in untreated years and relative to contemporaneous changes in other migpumas, holding area-year controls and fixed effects constant.
\begin{equation}
    \label{eq:regpop}
    Y_{my} = \beta'_{0} + \beta'_{1}\text{active agreement}_{my} + \alpha'_{1}X_{my} +  \alpha'_{2}W_{m} + \alpha'_{3}Z_{y} + \epsilon_{my} 
\end{equation}

To identify the effect of gaining and losing a 287(g) agreement in your local area, we implement a staggered difference-in-difference approach detailed in Equation \ref{eq:joindid}. The $\text{gains treatment}_{m}$ ($\text{loses treatment}_{m}$ ) indicator is assigned at the migpuma level and the $\text{post gain}_{my}$ ($\text{post loss}_{my}$) indicator is a dummy for the focus year being greater or equal than the year in which the treatment was gained (lost) in the focus migpuma. $\beta''_{1}$ captures the effect of gaining 287(g) treatment and $\beta''_{2}$ captures the effect of gaining treatment. The comparison group includes the never treated; those that will lose treatment, but before they lose it; and those that will gain treatment, but before they gain it. This setup allows us to identify the effect of obtaining or losing the treatment allowing for both events to be different from each other. Because ACS population counts are more sensitive to changes in inflows than to changes in outflows due to the sampling design, this specification is better powered to detect increases in the target population. The identifying assumption expects that treated migpumas, had not gained or lost treatment, would have perceived similar trends in totals of target population as never treated migpumas. I will evaluate this parallel trends assumption through an event study design. 
 \begin{equation}
    \label{eq:joindid}
    \begin{split}
    Y_{my} = \beta''_{0} + \beta''_{1}\text{gains treatment}_{m}*\text{post gain}_{my} + \beta''_{2}\text{loses treatment}_{m}*\text{post loss}_{my} \\
    +  \alpha''_{1}W_{m} + 
    \alpha''_{1}X_{m} +  \alpha''_{2}W_{m} + \alpha''_{3}Z_{y} + \epsilon_{my} 
    \end{split}
\end{equation}

Though we focus on a population specifically targeted by 287(g) deportations, I also evaluate effects on a placebo population that is not affected by this policy: U.S. born citizens of Hispanic origin with similar non-immigration characteristics (males, ages 18-39, with a high school degree or less, not currently married). Evaluating a placebo population would allow us to discern if effects in the total population are distinguishable from those that are not affected. As a way to evaluate spillover effects, I can also evaluate a population that \textit{should not} be affected by the policy: foreign-born U.S. citizens from Hispanic descent, with similar non-immigration characteristics (males, ages 18-39, with a high school degree or less, not currently married). 


    
 \section{Results}

Table \ref{tab:regprob} evaluates Equation \ref{eq:regpr} in a sample restricted to the target population in Columns 1 and 2 and restricted to the placebo population in Columns 3-4. Through this setup we find that the target population in a migpuma with an active 287(g) agreement is more likely to have moved to their current place of residence within the last year by 2.6 percentage points (or 21.25\% more relative to untreated migpumas) this results remain within this level after controlling for individual covariates. In the placebo population we find a negligible effect of the treatment in the probability of having moved to the current migpuma. The saliency of these results indicate that the composition of the target population in treated migpuma is different from comparable untreated migpumas. The direction and magnitude of these results should be taken with caution as they are not accounting for an evolving population size within a migpuma. It is possible that treated localities are observing changes in the total target population that have lived in the locality for over a year, thus driving the probability of moving upwards. We evaluate these hypothesis in Table \ref{tab:regpop}, in which we evaluate changes in size of target and placebo population at the migpuma-year level (Equation \ref{eq:regpop}). In this design, I find that the size target population in a migpuma with an active 287(g) agreement declines by 12.56\% and by 13.48\% when including controls at the migpuma-year level, signidicant at the 1 percent level. Meanwhile, the size of the placebo population decreases at a much smaller and statistically insignificant rates, 1.6\% and by 2.4\% when including controls.

I next hone into the effects of gaining and losing a 287(g) agreement by implementing a difference-in-difference design specified in Equation \ref{eq:joindid}. In Table \ref{tab:didpop}, the target population decreases by 5.9\% (Column 2) when gaining a 287(g) agreement though this effect presents large standard errors and are close in magnitude to the observed effect in the size and direction of the placebo population 7.7\% (Column 4). The effect of losing a 287(g) agreement in a migpuma can be associated with a 17.6\% increase of total target population, significant at the 1 percent level; meanwhile, the effect in the placebo is negligibly small (less than 0.5\%) and have large standard errors. When a locality gains a 287(g), we may expect a decrease in in-migration and/or an increase in out-migration as the saliency of the risk in deportation increases; the direction of our results are consistent with this intuition. On the other hand, when a locality loses a 287(g), we may expect an increase in in-migration and/or decrease in out-migration; our results concurs with this theory and as expected, an increase in population can be more easily identified through our setup.

To first test the parallel trends assumption through event studies\footnote{The regression design for this event studies includes both gaining and losing in the same regression but I plot these effects separately to ease interpret ability.}. In Panel (a), we can see that prior to gaining a 287(g) agreement the changes in the size of the treated population was closer to zero but generally lower than the reference period (a year before gaining treatment). After gaining treatment, the size of the population declines at higher rates starting one year after the agreement was enacted, reaching higher rates two years post, and increasing three years post. The trends in the post-period are close in direction and magnitude to those for the size of the placebo population (Panel (c)) up to two years after the enactment and keep on trending down three years after. 


increases mobility within county, across counties, and across states, with a higher significance level at county level. These results become stronger and more significant when applying propensity weights. In the placebo population, we observe mostly negative effects and not-statistically significant effects of 287(g) programs in mobility.


These results suggest there are differential effects in migration for the non-citizen targeted population, distinct from similar citizens. However, the direction of the effect has the opposite of the sign expected. I initially hypothesized that areas in which 287(g) is currently enacted, targeted immigrants would be less likely to move to areas where these agreements have been recently enacted. We are seeing that areas with active MOAs are more likely to have newcomers from the targeted population. A possible explanation for this, is that deporting immigrants in this group creates job vacancies and may potentially increase wages if demand for low-skilled Hispanic workers is not met, thus immigrants move to these areas in pursue of better economic opportunities. To validate this new hypothesis we must also check economic outcomes.


\newpage 
\bibliography{bib_287.bib}

\newpage
\section{Tables and figures}    


%%%%%% TABLES

%% table 1
\begin{table}[H]
\centering
\caption{ICE Removals from a 287(g) apprehension}
\label{tab:removals}
\adjustbox{ width=0.5\linewidth}{
    \begin{tabular}{rrrll}
\toprule
\toprule
\multicolumn{1}{l}{Characteristics}                         & Removals & Share  &  &  \\\midrule 
\multicolumn{1}{l}{Gender}                                  &          &        &  &  \\Male  & 87,923   & 0.9587 &  &  \\Female & 3,790    & 0.0413 &  &  \\\multicolumn{1}{l}{Age group}                               &          &        &  &  \\0-17  & 82       & 0.0009 &  &  \\18-24 & 17,911   & 0.1953 &  &  \\25-29 & 21,588   & 0.2354 &  &  \\30-34 & 19,568   & 0.2134 &  &  \\35-39 & 14,631   & 0.1595 &  &  \\40-44 & 9,040    & 0.0986 &  &  \\45-49 & 4,985    & 0.0544 &  &  \\50-54 & 2,384    & 0.0260 &  &  \\55-59 & 1,004    & 0.0109 &  &  \\60-64 & 365      & 0.0040 &  &  \\65-69 & 115      & 0.0013 &  &  \\70-74 & 32       & 0.0003 &  &  \\75+   & 9        & 0.0001 &  &  \\\multicolumn{1}{l}{Latin American or} & 90,235   & 0.9839 &  &  \\\multicolumn{1}{l}{Caribbean Citizenship} &   &  &  &  \\\multicolumn{1}{l}{Citizenship}                             &          &        &  &  \\Mexico & 68,038   & 0.7418 &  &  \\Guatemala & 8,042    & 0.0877 &  &  \\Honduras & 6,618    & 0.0722 &  &  \\El Salvador & 5,300    & 0.0578 &  &  \\Nicaragua & 329      & 0.0036 &  &  \\Brazil & 306      & 0.0033 &  &  \\\multicolumn{1}{l}{Seriousness Level of Conviction}         &          &        &  &  \\No Conviction & 16,270   & 0.1774 &  &  \\Level 1 Crime & 25,620   & 0.2793 &  &  \\Level 2 Crime & 9,065    & 0.0988 &  &  \\Level 3 Crime & 40,759   & 0.4444 &  & \bottomrule
\bottomrule
\end{tabular}
    
}

\justifying

\begin{spacing}{1}
\begin{footnotesize}
\noindent \textit{Notes:} This table summarizes characteristics of foreign citizens removed by ICE following a 287(g) apprehension from 2013-2019. Latin American citizenship is identified through the top 50 reported countries. Citizenship reports only the leading 5 countries of citizenship. Source: TRAC Immigration Tools \citep{trac24}.
\end{footnotesize}
\end{spacing}
\end{table}


%% table 2
\newpage
\begin{table}[H]
\begin{center}

\caption{Summary statistics by exposure and target group}
\label{tab:balance}
\adjustbox{ width=0.7\linewidth}{
    \begin{tabular}{lccc}
\toprule
\toprule
 & \multicolumn{3}{c}{Target population} & \multicolumn{3}{c}{Migpuma}  \\
 & Treated & Untreated & Difference   \\
 & (1) & (2) & (3)  \\
\midrule 
 \textbf{Mobility} & & &   \\
 Moved migpuma  & 0.12  & 0.13  & -0.00 \\
 & (0.01) & (0.00) & (0.01)\\
 Moved state  & 0.10  & 0.09  & 0.01 \\
 & (0.00) & (0.00) & (0.01)\\
 Moved from abroad  & 0.09  & 0.07  & 0.02*** \\
 & (0.00) & (0.00) & (0.01)\\
 \textbf{Demographics and education} & & &   \\
 Age  & 26.31  & 26.14  & 0.17 \\
 & (0.09) & (0.08) & (0.12)\\
 Number of children  & 0.21  & 0.21  & -0.00 \\
 & (0.01) & (0.01) & (0.01)\\
 Race: White  & 0.64  & 0.53  & 0.10*** \\
 & (0.01) & (0.01) & (0.01)\\
 Race: Black  & 0.01  & 0.02  & -0.01*** \\
 & (0.00) & (0.00) & (0.00)\\
 High School  & 0.47  & 0.45  & 0.01 \\
 & (0.01) & (0.01) & (0.01)\\
 Poor English  & 0.69  & 0.67  & 0.02* \\
 & (0.01) & (0.01) & (0.01)\\
 In School  & 0.09  & 0.08  & 0.01 \\
 & (0.00) & (0.00) & (0.01)\\
 \textbf{Employment and housing} & & &   \\
 Employed  & 0.84  & 0.84  & 0.01 \\
 & (0.01) & (0.00) & (0.01)\\
 Weeks worked  & 40.01  & 41.39  & -1.37 \\
 & (0.78) & (0.65) & (1.02)\\
 Usual weekly hours worked  & 35.19  & 35.30  & -0.11 \\
 & (0.24) & (0.20) & (0.31)\\
 Wage income  & 21,500.92  & 21,852.45  & -351.53 \\
 & (251.16) & (237.65) & (345.80)\\
 Owns a home  & 0.16  & 0.15  & 0.01 \\
 & (0.01) & (0.00) & (0.01)\\
 Rent price  & 1,008.73  & 1,035.85  & -27.11** \\
 & (7.41) & (8.04) & (10.93)\\
 Mortgage price  & 872.33  & 910.53  & -38.20 \\
 & (31.73) & (27.31) & (41.88)\\
Sample size & 5,963 & 8,151 & \\
\bottomrule
\bottomrule
\\
\end{tabular}
}
\end{center}

\justifying

\begin{spacing}{1}
\begin{footnotesize}
\noindent \textit{Notes:} This table summarizes main characteristics by exposure and population type. Columns 1-2 and 5-6 restrict the sample to immigrants likely targeted by 287(g) agreements: foreign-born non-citizen, Hispanic, male, of ages 18-39, with High School degree or less, that immigrated after 2007, have been in the US for 10 years or less, and are not currently married. Columns 3-4 and 7-8 restricts the sample to comparable individuals, meeting all criteria from the targeted sample except they must be US-born citizens (thus no immigration time requirements). Odd-numbered columns restrict the sample to (untreated) zero exposure counties and even-numbered columns restrict the sample to non-zero exposure counties. Columns 1-4 apply probability weights as extracted from ACS, while columns 5-8 apply weights obtained from propensity score matching.
\end{footnotesize}
\end{spacing}
\end{table}



%% table 3
\newpage
\begin{table}[H]
\centering
\caption{Net effect of 287(g) on probability of mobility among targeted and placebo populations}
\label{tab:regprob}
\adjustbox{ width=\linewidth}{
    \input{output/final/prob_in_migration}
}

\justifying
\begin{spacing}{1}
\begin{footnotesize}


\noindent \textit{Notes:} This table compares . p $<$ 0.01 ***, p $<$ 0.05 **, p $<$0.1 *. 

\end{footnotesize}
\end{spacing}
\end{table}



%% table 4
\newpage
\begin{table}[H]
\centering
\caption{Net effect of 287(g) on log total targeted and placebo populations}
\label{tab:regpop}
\adjustbox{ width=\linewidth}{
    \begin{tabular}{lcccc}
\toprule
\toprule
 & \multicolumn{2}{c}{Target population} & \multicolumn{2}{c}{Placebo population}  \\
 Log population & (1) & (2)  & (3) & (4)  \\
\midrule 
 Treated migpuma & -0.1357** & -0.1603** & -0.0170 & -0.0456 \\
 & (0.0634) & (0.0650) & (0.0567) & (0.0528) \\
\\
 Controls &  & X &  & X \\
 \textit{R2} & 0.9219 & 0.9236 & 0.9498 & 0.9515  \\
 Untreated pop size & 1,013 & 1,013 & 7,319 & 7,319  \\
Sample Size & 1,505 & 1,505 & 1,505 & 1,505  \\
\bottomrule
\bottomrule
\\
\end{tabular}
}

\justifying
\begin{spacing}{1}
\begin{footnotesize}


\noindent \textit{Notes:} This table compares . p $<$ 0.01 ***, p $<$ 0.05 **, p $<$0.1 *. 

\end{footnotesize}
\end{spacing}
\end{table}





\newpage
\begin{table}[H]
\centering
\caption{Difference-in-difference for log total targeted and placebo populations by gainer and loser status, separate regression}
\label{tab:didpop}
\adjustbox{ width=\linewidth}{
    \begin{tabular}{lcccc}
\toprule
\toprule
 & \multicolumn{2}{c}{Target population} & \multicolumn{2}{c}{Placebo population}  \\
Log population & (1) & (2)  & (3) & (4) \\
\midrule 
 Gain treatment & -0.2050 & -0.2284* & 0.0518 & 0.0787 \\
 & (0.1249) & (0.1285) & (0.1021) & (0.1012) \\
 Lose treatment & 0.0696 & 0.0583 & -0.1340 & -0.1354 \\
 & (0.0617) & (0.0602) & (0.1538) & (0.1509) \\
\\
 Controls &  & X &  & X \\
 \textit{R2} & 0.5949 & 0.6038 & 0.7401 & 0.7455  \\
 Untreated pop size & 832 & 832 & 1,483 & 1,483  \\
Sample Size & 3,487 & 3,487 & 3,487 & 3,487  \\
\bottomrule
\bottomrule
\\
\end{tabular}
}

\justifying
\begin{spacing}{1}
\begin{footnotesize}


\noindent \textit{Notes:} This table compares . p $<$ 0.01 ***, p $<$ 0.05 **, p $<$0.1 *. 

\end{footnotesize}
\end{spacing}
\end{table}



%%%%% FIGURES
%% figure 1
\newpage
\begin{figure}[H]
\centering
\caption{Yearly count and geographic distribution of MIGPUMAs with active 287(g) agreements}
\label{fig:agreement_count}
\begin{center}
\adjustbox{width=\linewidth}{
    \includegraphics[width=\linewidth]{output/final/bar_active_agreements.png}
}
\end{center}
\justifying
\begin{spacing}{1}
\begin{footnotesize}
\noindent \textit{Notes:} This graph shows yearly trends of MIGPUMAs with active agreements from 2012-2019.
\end{footnotesize}
\end{spacing}
\end{figure}


%% figure 2 
\begin{landscape}
\begin{figure}[H]
\centering
\caption{Event study of log total targeted and placebo populations by gainer and loser status}
\label{fig:didpop}

\begin{minipage}{0.44\linewidth}
\centering
\adjustbox{width=\linewidth}{\includegraphics{output/final/logtargetpop_gain_estudy.png}}
\end{minipage}
\begin{minipage}{0.44\linewidth}
\centering
\adjustbox{width=\linewidth}{\includegraphics{output/final/logtargetpop_lost_estudy.png}}
\end{minipage}
\begin{minipage}{0.44\linewidth}
\centering
\adjustbox{width=\linewidth}{\includegraphics{output/final/logplacebopop_gain_estudy.png}}
\end{minipage}
\begin{minipage}{0.44\linewidth}
\centering
\adjustbox{width=\linewidth}{\includegraphics{output/final/logplacebopop_lost_estudy.png}}
\end{minipage}

\justifying
\begin{spacing}{1}
\begin{footnotesize}
\noindent \textit{Notes: This graph shows the density of the indicator for migpuma ever exposed.}
\end{footnotesize}
\end{spacing}

\end{figure}

\end{landscape}




\section{Appendix}
%% figure 1
\begin{figure}[H]
\centering
\caption{Density of propensity score}
\label{fig:prop_score}
\begin{center}
\adjustbox{width=\linewidth}{
    \includegraphics[width=\linewidth]{output/final/prop_score.png}
}
\end{center}
\justifying
\begin{spacing}{1}
\begin{footnotesize}
\noindent \textit{Notes:} This graph shows the density of the indicator for migpuma ever exposed.
\end{footnotesize}
\end{spacing}
\end{figure}





\end{document}
